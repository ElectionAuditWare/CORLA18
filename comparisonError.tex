\section{Comparison audits of a tolerable overstatement in votes}
\label{sec:comparisonError}

We consider auditing in a single stratum to test whether the overstatement of any margin
(in votes) exceeds some fraction $\lambda$ of the overall margin $V_{w\ell}$ between
reported winner $w$ and reported loser $\ell$.
If the stratum contains all the ballots cast in the contest, then for $\lambda = 1$, this 
would confirm the election outcome.
For stratified audits, we might want to test other values of $\lambda$, as described above.

In Colorado, comparison audits have been ballot-level (i.e., batches consisting of a single
ballot). 
In this section, we derive a method for batches of arbitrary size, which might be useful
for Colorado to audit contests that include CVR counties and legacy counties.
We keep the \emph{a priori} error bounds tighter than the ``super-simple'' 
method~\citep{stark10d}.
To keep the notation simpler, we consider only a single contest, but the 
MACRO approach \citep{stark09c,stark10d} trivially extends the result to 
auditing $C>1$ contests simultaneously.
The derivation is for plurality contests, including ``vote-for-$k$'' plurality contests.
Majority and super-majority contests such as bond measures are a minor 
modification~\citep{stark08a}.\footnote{%
  So are some forms of preferential and approval voting, such as Borda count, and
  proportional representation contests, such as D'Hondt~\citep{starkTeague14}.
  Changes for IRV/STV are more complicated.
}

\subsection{Notation}
\begin{itemize}
    \item  $N$ ballots were cast in all. (The contest might not appear on all $N$ ballots)
    \item  $\mathcal{W}$: the set of reported winners of the contest
    \item  $\mathcal{L}$: the set of reported losers of the contest
    \item  $n_p$: number of ballots in batch $p$
    \item  $v_{pi} \in \{0, 1\}$: the reported votes for candidate $i$ in batch $p$
    \item  $a_{pi} \in \{0, 1\}$: actual votes for candidate $i$ in batch $p$. 
                  If the contest does not appear in batch $p$, then $a_{pi} = 0$.
    \item  $V_{w\ell} \equiv \sum_{p=1}^N (v_{pw} - v_{p\ell})$: 
Reported margin in the stratum of reported winner $w \in \mathcal{W}$ over reported loser 
$\ell \in \mathcal{L}$.
    \item  $V$: smallest reported margin in the stratum among all $C$ contests audited using the same sample:
$V \equiv \min_{w \in \mathcal{W}, \ell \in \mathcal{L}} V_{w \ell}$
    \item  $\mu = V/N$: the ``diluted stratum margin,'' the margin in the stratum in votes divided by the total number of ballots in the stratum
    \item  $A_{w\ell} \equiv \sum_{p=1}^N (a_{pw} - a_{p\ell})$: 
actual margin in the stratum of reported winner $w \in \mathcal{W}$ over reported loser 
$\ell \in \mathcal{L}$
\end{itemize}
If the contest is entirely contained in the stratum, then
the reported winners of the contest are the actual winners if
$$ 
   \min_{w \in \mathcal{W}, \ell \in \mathcal{L}} A_{w\ell} > 0.
$$
Here, we address the case that the contest may include a portion outside the stratum.
To combine independent procedures in different strata, it is convenient
to be able to test whether the net error in a stratum exceeds a given threshold.

We won't test that inequality directly.
Instead, we will test a condition that is sufficient but not necessary for the
inequality to hold, to get a computationally simple test that
is still conservative (the risk is not larger than its nominal value).

For every winner, loser pair $(w, \ell)$, we want to test
whether the overstatement error exceeds a fraction $\lambda$ of the overall margin 
$V_{w\ell}$,
that is, we want to establish
$$
   \sum_{p=1}^N (v_{pw}-a_{pw} - v_{p\ell} + a_{p\ell})/V_{w\ell} < \lambda.
$$
Now the maximum (over all winner, loser pairs) of sums
is not larger than the sum of maxima; that is,
$$
\max_{w \in \mathcal{W},  \ell \in \mathcal{L}}
   \sum_{p=1}^N (v_{pw}-a_{pw} - v_{p\ell} + a_{p\ell})/V_{w\ell}
   \le
  \sum_{p=1}^N  \max_{w \in \mathcal{W},  \ell \in \mathcal{L}} 
  (v_{pw}-a_{pw} - v_{p\ell} + a_{p\ell})/V_{w\ell}.
$$
Hence, if 
$$
\sum_{p=1}^N  \max_{w \in \mathcal{W},  \ell \in \mathcal{L}} 
  (v_{pw}-a_{pw} - v_{p\ell} + a_{p\ell})/V_{w\ell} < 1,
$$
all the reported outcomes must be correct.
Define 
$$
  e_p \equiv \max_{w \in \mathcal{W} \ell \in \mathcal{L}} (v_{pw}-a_{pw} - v_{p\ell} + a_{p\ell})/V_{w\ell}.
$$
Then the reported outcomes of all the contests must be correct if 
$$ 
  E \equiv \sum_{p=1}^N e_p < 1.
$$

To see that a different way, suppose that the outcome of one or more contests is wrong.
Then there is some contest $c$ and some reported (winner, loser) pair
$w \in \mathcal{W}, \ell \in \mathcal{L}$ for which

$$ 
   0 \ge A_{w\ell} = V_{w\ell} - (V_{w\ell} - A_{w\ell}) =
   V_{w\ell} - \sum_{p=1}^N (v_{pw} - a_{pw} - v_{p\ell} + a_{p\ell}),
$$
i.e.,
$$ 
\sum_{p=1}^N (v_{pw} - a_{pw} - v_{p\ell} + a_{p\ell}) \ge V_{w\ell}.
$$
Diving both sides by $V_{w\ell}$ gives
$$
\sum_{p=1}^N \frac{v_{pw} - a_{pw} - v_{p\ell} + a_{p\ell}}{V_{w\ell}} \ge 1.
$$
But
$$
\frac{v_{pw} - a_{pw} - v_{p\ell} + a_{p\ell}}{V_{w\ell}}
\le 
\max_{w \in \mathcal{W}, \ell \in \mathcal{L}} \frac{v_{pw}-a_{pw} - v_{p\ell} + a_{p\ell}}{V_{w\ell}}
$$
$$
\le 
\max_{w \in \mathcal{W}, \ell \in \mathcal{L}} \frac{v_{pw}-a_{pw} - v_{p\ell} + a_{p\ell}}{V_{w\ell}}
= e_p,
$$
so if the outcome is wrong, $E = \sum_p e_p \ge 1$.
Thus a risk-limiting audit can rely on testing whether $E \ge 1$.
If the hypothesis $E \ge 1$ can be rejected at significance level $\alpha$,
we can conclude that all the reported outcomes are correct.

Testing whether $E \ge 1$ would require a very large sample if we knew nothing at
all about $e_p$ without auditing ballot $p$: a single large value of $e_p$ could make
$E$ arbitrarily large.
Fortunately, there is an \emph{a priori} upper bound for $e_p$.
Whatever the reported votes $v_{pi}$ are on ballot $p$, we can find the
potential values of the actual votes $a_{pi}$ that would make the
error $e_p$ largest, because $a_{pi}$ can only be zero or one:
$$
    \frac{v_{pw}-a_{pw} - v_{p\ell} + a_{p\ell}}{V_{w\ell}} \le 
    \frac{v_{pw}- 0 - v_{p\ell} + 1}{V_{w\ell}}.
$$
Hence,
$$
    e_p \le \max_{w \in \mathcal{W}, \ell \in \mathcal{L}} 
    \frac{v_{pw} - v_{p\ell} + 1}{V_{w\ell}} \equiv \tilde{u}_p.
$$

Knowing that $e_p \le \tilde{u}_p$ might let us conclude reliably that $E < 1$
by examining only a small fraction of the ballots---depending on the 
values $\{ \tilde{u}_p\}_{p=1}^N$ and on the values of $\{e_p\}$ for the audited ballots.

To make inferences about $E$, it is helpful to work with the \emph{taint} $t_p \equiv \frac{e_p}{\tilde{u}_p} \le 1$.
Define $\tilde{U} \equiv \sum_{p=1}^N \tilde{u}_p$.
Suppose we draw ballots at random with replacement, with probability $\tilde{u}_p/\tilde{U}$
of drawing ballot $p$ in each draw, $p = 1, \ldots, N$.
(Since $\tilde{u}_p \ge 0$, these are all positive numbers, and they sum to 1,
so they define a probability distribution on the $N$ ballots.)

Let $T_j$ be the value of $t_p$ for the ballot $p$ selected in the $j$th draw.
Then $\{T_j\}_{j=1}^n$ are IID, $\mathbb{P} \{T_j \le 1\} = 1$, and
$$
  \mathbb{E} T_1 = \sum_{p=1}^N \tilde{u}_p/\tilde{U} t_p =
  \frac{1}{\tilde{U}}\sum_{p=1}^N \tilde{u}_p \frac{e_p}{\tilde{u}_p} = 
  \frac{1}{\tilde{U}} \sum_{p=1}^N e_p = E/\tilde{U}.
$$
Thus $E = \tilde{U} \mathbb{E} T_1$. 

So, if we have strong evidence that
$\mathbb{E} T_1 < 1/\tilde{U}$, we have
strong evidence that $E < 1$.

This approach can be simplified even further by noting that $\tilde{u}_p$ has
a simple upper bound that does not depend on any $v_{pi}$.
At worst, the CVR for ballot $p$ shows a vote for the "least-winning" apparent winner of the contest with the smallest margin, but a hand interpretation shows a vote for the runner-up in
that contest.
Since $V_{w\ell} \ge V$ and $0 \le v_{pi} \le 1$,
$$ 
   \tilde{u}_p =  \max_{w \in \mathcal{W}, \ell \in \mathcal{L}} 
    \frac{v_{pw} - v_{p\ell} + 1}{V_{w\ell}}
    \le  \max_{w \in \mathcal{W}, \ell \in \mathcal{L}} 
    \frac{1 - 0 + 1}{V_{w\ell}}
    \le \frac{2}{V}.
$$

Thus, if we define $u_p \equiv 2/V$ and
sample ballots at random with probability proportional to $u_p$, in fact
we will sample ballots with \emph{equal} probability.
Define
$$ 
   U \equiv \sum_{p=1}^N \frac{2}{V} = \frac{2N}{V} = 2/\mu
$$
and re-define $t_p \equiv e_p/u_p$ (rather than $e_p/\tilde{u}_p$);
let $T_j$ be the value of $t_p$ for the ballot selected at random in the
$j$th draw, as before.
Then still $\{T_j\}_{j=1}^n$ are IID, $\mathbb{P} \{T_j \le 1\} = 1$,
and
$$
  \mathbb{E} T_1 = \sum_{p=1}^N \frac{u_p}{U} t_p =
  \frac{1}{U}\sum_{p=1}^N u_p \frac{e_p}{u_p} = 
  \frac{1}{U} \sum_{p=1}^N e_p = E/U = \frac{\mu}{2} E,
$$
i.e., 
$$ 
    E = \frac{2}{\mu}\mathbb{E} T_1.
$$
So, if we have evidence that $\mathbb{E} T_1 < \mu/2 = 1/U$, we have evidence that 
$E < 1$.