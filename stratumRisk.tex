\subsubsection{Combining stratum-level risk limits}\label{sec:stratumRisk}
We audit to test the two hypotheses $\{\omega_{w\ell,s} \ge \lambda_s V_{w\ell}\}_{s=1}^2$, 
independently for the two strata.
If we reject \emph{both} hypotheses, we conclude that the contest outcome is correct;
otherwise, we manually re-tabulate the contest in one or both strata, depending on the
audit rules.
Those rules matter:
generally, the two audits will need to be conducted to smaller risk limits individually than the desired
risk limit for the contest as a whole, unless proceeding to a full hand tabulation in one stratum
automatically triggers a full hand tabulation in the other stratum.

Recall that the samples are drawn independently from the two strata.
Pick $\alpha_1, \alpha_2 \in (0,\alpha)$.
(Below we discuss the choice further.)
Also pick $\lambda_1$.
Then if $\omega_{w\ell,1} < \lambda_1 V_{w\ell}$ and 
$\omega_{w\ell,2} < \lambda_2 V_{w\ell}$,
the outcome is correct.
We audit stratum $s$ to test the hypothesis $\omega_{w\ell,s} \ge \lambda_s V_{w\ell}$ 
with risk limit $\alpha_s$,
as if it were its own election.
We want to know the relationship between those two stratum-level ``risks'' and the 
overall risk that the audit will not correct the outcome if the outcome is wrong.
That depends in part on what we do if the audit in a given stratum leads to a full manual
tally of that stratum.

Consider a few scenarios.
The outcome is certainly correct if both net overstatements are less than their 
respective thresholds. 
For the outcome to be wrong, one or both strata need to have net overstatement
$\omega_{w\ell,s}$
greater than its corresponding threshold $\lambda_s V_{w\ell}$.
If $\omega_{w\ell,1} + \omega_{w\ell,2} \ge V_{w\ell}$, then $\omega_{w\ell,1}\ge \lambda_1V_{w\ell}$
or $\omega_{w\ell,2}\ge \lambda_2V_{w\ell}$, or both.
If the allocated overstatement is exceeded in only one stratum, $s$, then the chance that the 
stratum will be fully hand counted is at least $1-\alpha_s \ge 1- \alpha$.

If both $\omega_{w\ell,1} \ge \lambda_1V_{w\ell}$
and $\omega_{w\ell,2} \ge \lambda_2V_{w\ell}$, then the chance both are fully tabulated is
$1-(1-\alpha_1)(1-\alpha_2)$, since the audit samples in the two strata are independent.

What should we do if the audit leads to a full tally in one stratum?
We consider two options.
The simpler is to automatically require a full hand count of the other stratum, 
to set the record straight.
If the audit uses this rule, then we can take $\alpha_1 = \alpha_2 = \alpha$, and the procedure will have
risk limit $\alpha$.

A second approach is to adjust the contest margin for the results of the manual tally in the
fully counted stratum (call the stratum $t$), and continue to audit in the other, 
but against the overall margin, adjusted for the ``known'' tally in the stratum that had 
been counted: we test against the share $V_{w\ell} - A_{w\ell,t} \equiv \lambda_s' V_{w\ell}$, rather than 
against the share $\lambda_s V_{w\ell}$.
Then to reject the null hypothesis in that stratum is to conclude that the overall outcome is still correct.

The statistical wrinkle is that adjusting for the manual tally in the hand-counted stratum 
changes the hypothesis being tested in the remaining
stratum in a way that is itself random:
whether the original null or a new null is tested depends on what the sample in the other stratum
finds.
However, if the hypothesis changes, there's only one value possible for $\lambda_s'$---which
depends on the reported margin and the count in the other stratum---but it's unknown 
until the other stratum count is known.

The solution is through conditioning. 
The samples in the two strata are independent. 
Think of the overall procedure as concluding that the outcome is correct without a full
hand count in both strata if:

\begin{itemize}
   \item the original hypotheses are rejected in both strata (neither stratum is fully hand tabulated)
   \item the hypothesis is not rejected in one stratum $s$; the threshold $\lambda_t$
            is adjusted in the other stratum, and the hypothesis that the overstatement error
            in that stratum is greater than the new limit, $\lambda_t' V_{w\ell}$ is rejected. 
\end{itemize}

Suppose that the outcome is incorrect. 
Then in at least one stratum $s$, $\omega_{w\ell,s}\ge \lambda_s V_{w\ell}$.
The chance that the audit leads to a full hand tabulation in that stratum is at least $1-\alpha_s$,
in which case the ``correct'' count for that stratum will become known.

What happens in the other stratum?
We adjust the margin (or tolerance for overstatement error) and keep auditing.
The value of $\lambda_t'$ is fixed, but unknown before the audit starts.
Consider the conditional probability that a sequential test would reject the hypothesis that the margin is less than $\lambda_t' V_{w\ell}$, given that the other stratum, $s$, is tabulated by hand.
Because the tests in the two strata are independent, that is the unconditional probability
that auditing stratum $t$ against the error tolerance $\lambda_t' V_{w\ell}$ would lead to a full
hand tabulation. 
If we are using a sequential test in the remaining stratum $t$, the chance that the audit will
go to a full hand count in stratum $t$ is at least $1-\alpha_t$ if the (new) null is true (i.e.,
if the outcome is incorrect). 
The conditioning just delays looking at the value of the test statistic for the new null hypothesis; waiting does not increase the overall chance of incorrectly rejecting the null, 
because the test is legitimately sequential.

Hence, for this procedure, the chance that there will be a full hand count in both strata is at least 
$(1-\alpha_s)(1-\alpha_t)$ if the outcome is incorrect,
even if the probability were zero that both of the original audits would proceed to a full hand count.
The overall risk limit is thus not larger than $1 - (1-\alpha_s)(1-\alpha_t)$.

Suppose we want the overall risk limit to be 5\%. 
If we use a risk limit of 4\% in the no-CVR stratum and a risk limit of 1.04\% in the CVR stratum,
the risk limit will be $1 - 0.96\times 0.9896 < 0.05$.

