\subsubsection{Combining stratum-level risk limits}\label{sec:stratumRisk}
We audit to test the two hypotheses $\{\omega_{w\ell,s} \ge \lambda_s V_{w\ell}\}_{s=1}^2$, 
independently for the two strata.
If we reject \emph{both} hypotheses, we conclude that the contest outcome is correct;
otherwise, we manually re-tabulate the contest in one or both strata, depending on the
audit rules.
Those rules matter:
the two audits might need to be conducted to smaller risk limits individually than the desired
risk limit for the contest as a whole.

Recall that the samples are drawn independently from the two strata.
Pick $\alpha_1, \alpha_2 \in (0,\alpha)$.
(Below we discuss the choice further.)
Also pick $\lambda_1$.
Then if $\omega_{w\ell,1} < \lambda_1 V_{w\ell}$ and 
$\omega_{w\ell,2} < \lambda_2 V_{w\ell}$,
the outcome is correct.
We audit each stratum $s$ to test the hypothesis $\omega_{w\ell,s} \ge \lambda_s V_{w\ell}$ 
at risk limit $\alpha_s$,
as if it were its own election.
The audits can be conducted at the same time or sequentially; there is no coordination
between the audits unless one of them leads to a full hand count but the other does not:
see below.

We want to know the relationship between those two stratum-level ``risk limits'' $\alpha_1$ and
$\alpha_2$ and the 
overall risk that the audit will not correct the outcome if the outcome is wrong.
The overall risk depends on what we do if the audit in one stratum leads to a full manual
tally of that stratum.

Here are the possibilities.
The outcome is certainly correct if, in both strata, the net overstatement in the stratum is less
than its threshold $\lambda_s V_{w\ell}$.
For the outcome to be wrong, one or both strata need to have net overstatement
$\omega_{w\ell,s}$
greater than its corresponding threshold $\lambda_s V_{w\ell}$.
That is, if $\omega_{w\ell,1} + \omega_{w\ell,2} \ge V_{w\ell}$, then $\omega_{w\ell,1}\ge \lambda_1V_{w\ell}$
or $\omega_{w\ell,2}\ge \lambda_2V_{w\ell}$, or both.
If the threshold overstatement is exceeded in only one stratum, $h$, then the chance that the 
stratum will be fully hand counted is at least $1-\alpha_h \ge 1- \alpha$.

If both $\omega_{w\ell,1} \ge \lambda_1V_{w\ell}$
and $\omega_{w\ell,2} \ge \lambda_2V_{w\ell}$, then the chance both 
are completely tabulated by hand is at least
$(1-\alpha_1)(1-\alpha_2)$, since the audit samples in the two strata are independent.

What should we do if the audit leads to a full tally in one stratum, $h$,
but the other audit has not led to a full tabulation, because it 
has not started, because it is still underway, or because it terminated without
a full hand tally?
We consider two options.
The simpler is to automatically require a full hand count of the other stratum. 
If the audit uses this rule, then we can take $\alpha_1 = \alpha_2 = \alpha$, 
and the procedure will have risk limit~$\alpha$.

A second approach is to adjust the contest margin for the results of the manual 
tally in the fully counted stratum, $h$, and to audit the other
against the overall margin, adjusted for the known manual tally $A_{w\ell,h}$
in the stratum $h$ that has been fully hand tallied:
we test against the threshold 
$V_{w\ell} - A_{w\ell,h} \equiv \lambda_t' V_{w\ell}$, rather than 
against its (original) threshold $\lambda_t V_{w\ell}$.
Then to reject the new null hypothesis in stratum $t$ is to conclude that the 
overall outcome is correct.

If and when the hypothesis in stratum $t$ changes, the audit
in that stratum might be able to stop on the basis of the data already observed;
it might need to continue; or---if it had stopped based on the original threshold
$\lambda_t V_{w\ell}$, it might need to examine more ballots, possibly
continuing to a full hand tally.

The statistical wrinkle is that adjusting for the manual tally in the hand-counted 
stratum $h$
changes the hypothesis being tested in the other stratum $t$
in a way that is itself random:
whether the original null $\omega_{w\ell,s} \ge \lambda_t V_{w\ell}$ is tested
or the new null $\omega_{w\ell,s} \ge \lambda_t' V_{w\ell}$ is tested depends on what the 
sample reveals in stratum $h$.
If the hypothesis does change, there is only one value possible for $\lambda_t'$---which
depends on the reported margin $V_{w\ell}$ and the count $A_{w\ell,h}$ in 
stratum $h$---but $\lambda_t'$ is unknown until $A_{w\ell,h}$ is known.

We assume that before any data are collected, the audit specifies two families of tests:
for each stratum $s$, a family of level-$\alpha_s$ tests of the null hypothesis that 
the overstatement in the stratum is greater than or equal to $c$, for all feasible values of $c$.
That is,
\beq
    \Pr \{ \mbox{reject hypothesis that } \omega_{w\ell,s} \ge 
    c_s || \omega_{w\ell,s} \ge c_s \} \le \alpha_s,
\eeq
for $s = 1, 2$, and all feasible $c_s$.
Moreover, we insist that the test depend on data only from ballots selected from its stratum.
Because the samples in the two strata are independent, for all feasible pairs $c_1, c_2$,
\begin{align} \label{eq:stratum_families}
    \Pr\{&\mbox{reject neither hypothesis } \omega_{w\ell,s} \ge c_s, \;\; s=1, 2 ||
       \omega_{w\ell,s} \ge c_s  \mbox{ for both } s=1, 2 \} \nonumber \\ 
       &= \prod_{s=1}^2 1 - \Pr \{ \mbox{reject hypothesis that } \omega_{w\ell,s} \ge c_s || \omega_{w\ell,s} \ge c_s \} \nonumber \\
       & \ge (1-\alpha_1)(1-\alpha_2).
\end{align}

What is the chance that the audit leads to a full hand tabulation if the outcome is incorrect?
One way the audit can lead to a full hand tally is if it leads to a full count in one stratum, 
the null hypothesis in the other stratum is changed, and the audit in the second 
stratum then proceeds to a full manual tally.
(There are other ways the audit can lead to a full hand tally, for instance, if neither
null hypothesis is rejected, but this is one way.)

If the outcome is wrong, there is at least one stratum in which the overstatement 
$\omega_{w\ell,s}$ 
exceeds the threshold $\lambda_s V_{w\ell}$.
Let $h$ be one such stratum. 
Then the chance the audit in stratum $h$ leads to a full manual tally in that stratum
is at least $(1-\alpha_h)$.
If the audit leads to a full manual tally in stratum~$h$ and the overall outcome is wrong,
then the (new) null hypothesis in the other stratum, $t$ must be true.
If we started to audit that new hypothesis \emph{ab initio}, the chance that we would reject it
would be at most $\alpha_t$, so the chance the audit would lead to a full hand count 
of stratum $t$ is at least $1-\alpha_t$.
The question is whether ``changing hypotheses'' could make that chance smaller.
The inequality \ref{eq:stratum_families} shows that it cannot: for any feasible pair of
overstatements, $c = (c_1, c_2)$, if $\omega_{w\ell,1} \ge c_1$ and $\omega_{w\ell,2} \ge c_2$,
the chance that neither the hypothesis $\omega_{w\ell,1} \ge c_1$ nor the hypothesis 
$\omega_{w\ell,2} \ge c_2$ will be rejected is at least $(1-\alpha_1)(1-\alpha_2)$.

And therefore, for this procedure, the chance that there will be a full hand count in both strata is at least 
$(1-\alpha_1)(1-\alpha_2)$ if the outcome is incorrect,
even if the probability were zero that both of the original audits would proceed to a full hand count.
The overall risk limit is thus not larger than $1 - (1-\alpha_s)(1-\alpha_t)$.

As an example, suppose we want the overall risk limit to be 5\%. 
If we use a risk limit of 4\% in the no-CVR stratum and a risk limit of 1.04\% in the CVR stratum,
the risk limit will be $1 - 0.96\times 0.9896 < 0.05$.

