\section{Ballot-polling audits of a tolerable overstatement in votes}
\label{sec:ballotPollError}

\subsection{Tri-hypergeometric test using the diluted margin as the test statistic}

We consider a single stratum $s$, containing $N_s$ ballots.
We will sample individual ballots without replacement from stratum $s$.
Of the $N_s$ ballots,
$A_{w,s}$ have a vote for $w$ but not for $\ell$, $A_{\ell,s}$ have a vote for $\ell$ but not for $w$, and $A_{u,s} = N_s - N_{w,s} - N_{\ell,s}$ have votes for both $w$ and $\ell$ or neither $w$ nor $\ell$, including undervotes and invalid ballots.
We might draw a simple random sample of $n$ ballots ($n$ fixed ahead of time), or we might draw 
sequentially without replacement, so the sample size $B$ could be random, including stopping rules that depend on the data.\footnote{%
   Sampling with replacement leads to simpler arithmetic, but is not as efficient.
}

Regardless, we assume that, conditional on the attained sample size $n$, the ballots are a simple random sample of size $n$ from the $N_s$ ballots in the population.
In the sample, $B_w$ contain a vote for $w$ but not $\ell$, with $B_\ell$ and $B_u$ defined analogously.
Then, conditional on $B=n$, the joint distribution of
$(B_w, B_\ell, B_u)$ is tri-hypergeometric:

\begin{equation}
    \mathbb{P}_{A_{w,s}, A_{\ell,s}} \{ B_w = i, B_\ell = j \vert B=n \} = 
     \frac{ {N_{w,s } \choose i}{N_{\ell,s} \choose j}{N_s - A_{w,s} - A_{\ell,s} \choose n-i-j}}{{N_s \choose n}}.
\end{equation}

To test, we will condition on the event $B=n$. 
(In contrast, the BRAVO ballot-polling
method~\citep{lindemanEtal12}
conditions on $B_w+B_\ell = m$;
we might instead consider conditioning on $B=n$ and $B_w + B_\ell = m$.)
The test statistic will be the diluted sample margin, $D \equiv (B_w - B_\ell)/B$.
This is the sample difference in the number of ballots for the winner and for the loser, divided by the 
total number of ballots in the sample.
We want to test the compound hypothesis $A_{w,s} - A_{\ell,s} \le c$.
The value of $c$ is inferred from the definition
$\omega_{w\ell,s} \equiv V_{w\ell,s} - A_{w\ell,s} = V_{w,s} - V_{\ell,s} - (A_{w,s} -A_{\ell,s})$.
Thus,
$$
    c = V_{w,s} - V_{\ell,s} - \omega_{w\ell,s} = V_{w\ell,s} - \lambda_s V_{w\ell}.
$$
The alternative is the compound hypothesis 
$A_{w,s} - A_{\ell,s} > c$.\footnote{%
    To use Wald's Sequential Probability Ratio Test, we might pick a simple alternative instead, e.g.,
   $A_{w,s} = V_{w,s}$ and $A_{\ell,s} = V_{\ell,s}$, the reported values, assuming 
   $V_{w,s} - V_{\ell,s} > c$.
}
Hence, we will reject for large values of $D$.
Conditional on $B=n$, the event $D = (B_w - B_\ell)/B = d$ is the event $B_w - B_\ell = nd$.

Suppose we observe $D=d$.
The $p$-value of the simple hypothesis that there are $N_w$ ballots with
a vote for $w$ but not for $\ell$, $N_\ell$ ballots with a vote for $\ell$ but not for $w$, and $N - N_w - N_\ell$ ballots with votes for both $w$ and $\ell$ or neither $w$ nor $\ell$ (including undervotes and
invalid ballots) is

\begin{equation}
\begin{aligned}
   \mathbb{P}_{A_{w,s}, A_{\ell,s}, N_s} \left \{ D \geq d \vert B = n \right \} 
  & = \mathbb{P}_{A_{w,s}, A_{\ell,s}, N_s}\{ B_w - B_\ell \geq nd \vert B = n \} \\
   & = \sum_{(i, j) : i - j \geq nd; i+j \leq n; i, j\geq 0} 
         \frac{ {A_{w,s} \choose i}{A_{\ell,s} \choose j}{N_s - A_{w,s} - N_{\ell,s} \choose n-i-j}}{{N_s \choose n}}.
\end{aligned}
\end{equation}
Below we show empirically (but do not prove) that this tail probability is monotone increasing in $A_{w,s}$.

\note{Address numerical stability and ways to simplify the expression.}

The composite null hypothesis does not specify $A_{w,s}$ or $A_{\ell,s}$ separately, only 
that $A_{w,s} - A_{\ell,s} \le c$ for
some fixed, known $c$.
The (conditional) $p$-value of this composite hypothesis for $D=d$ is
\begin{equation}
  \max_{A_{w,s}, A_{\ell,s} \in \{0, 1, \ldots, N \}: A_{w,s} - A_{\ell,s} \le c, A_{w,s} + A_{\ell,s} \le N_s}
   \sum_{(i, j) : i - j \geq nd; i+j \leq n; i, j\geq 0} 
         \frac{ {A_{w,s} \choose i}{A_{\ell,s} \choose j}{N_s - A_{w,s} - A_{\ell,s} \choose n-i-j}}{{N_s \choose n}},
\end{equation}
wherever the summand is defined. 
(Equivalently, define ${m \choose k} \equiv 0$ if $k > m$, $k < 0$, or $m \le 0$.)
If the empirical result mentioned above is true, then finding the maximum is trivial; if not,
it is still at most a one-dimensional optimization problem.

\note{We could condition on $B$ and $B_w + B_\ell$, which would give us a binomial \textellipsis Worth trying? Any intuitive argument for why it would be better or worse? Might make using SPRT easier.}
