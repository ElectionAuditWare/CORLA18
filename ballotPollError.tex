\section{Ballot-polling audits of a tolerable overstatement in votes}
\label{sec:ballotPollError}

\subsection{Conditional hypergeometric test}

We consider a single stratum $s$, containing $N_s$ ballots.
We will sample individual ballots without replacement from stratum $s$.
Of the $N_s$ ballots,
$A_{w,s}$ have a vote for $w$ but not for $\ell$, $A_{\ell,s}$ have a vote for $\ell$ but not for $w$, and $A_{u,s} = N_s - N_{w,s} - N_{\ell,s}$ have votes for both $w$ and $\ell$ or neither $w$ nor $\ell$, including undervotes and invalid ballots.
We might draw a simple random sample of $n$ ballots ($n$ fixed ahead of time), or we might draw 
sequentially without replacement, so the sample size $B$ could be random.
For instance, the rule for determining $B$ could depend on the data.\footnote{%
   Sampling with replacement leads to simpler arithmetic, but is not as efficient.
}

Regardless, we assume that, conditional on the attained sample size $n$, the ballots are a simple random sample of size $n$ from the $N_s$ ballots in the population.
In the sample, $B_w$ ballots contain a vote for $w$ but not $\ell$, with $B_\ell$ and $B_u$ defined analogously.
Then, conditional on $B=n$, the joint distribution of
$(B_w, B_\ell, B_u)$ is tri-hypergeometric:

\begin{equation}
    \mathbb{P}_{A_{w,s}, A_{\ell,s}} \{ B_w = i, B_\ell = j \vert B=n \} = 
     \frac{ {A_{w,s } \choose i}{A_{\ell,s} \choose j}{N_s - A_{w,s} - A_{\ell,s} \choose n-i-j}}{{N_s \choose n}}.
\end{equation}

The test statistic will be the diluted sample margin, $D \equiv (B_w - B_\ell)/B$.
This is the sample difference in the number of ballots for the winner and for the loser, divided by the 
total number of ballots in the sample.
We want to test the compound hypothesis $A_{w,s} - A_{\ell,s} \le c$.
The value of $c$ is inferred from the definition
$\omega_{w\ell,s} \equiv V_{w\ell,s} - A_{w\ell,s} = V_{w,s} - V_{\ell,s} - (A_{w,s} -A_{\ell,s})$.
Thus,
$$
    c = V_{w,s} - V_{\ell,s} - \omega_{w\ell,s} = V_{w\ell,s} - \lambda_s V_{w\ell}.
$$
The alternative is the compound hypothesis 
$A_{w,s} - A_{\ell,s} > c$.\footnote{%
    To use Wald's Sequential Probability Ratio Test, we might pick a simple alternative instead, e.g.,
   $A_{w,s} = V_{w,s}$ and $A_{\ell,s} = V_{\ell,s}$, the reported values, provided 
   $V_{w,s} - V_{\ell,s} > c$.
}
Hence, we will reject for large values of $D$.
Conditional on $B=n$, the event $D = (B_w - B_\ell)/B = d$ is the event $B_w - B_\ell = nd$.

Suppose we observe $D=d$.
The test will condition on the event $B=n$ and $B_w +B_\ell = m$. 
(In contrast, the BRAVO ballot-polling
method~\citep{lindemanEtal12}
conditions only on $B_w+B_\ell = m$.)

Given $B=n$, all samples of size $n$ from the ballots are equally likely, by hypothesis.
Hence, in particular, all samples of size $n$ for which $B_w + B_\ell = m$ are equally likely.
There are ${A_{w,s}+A_{\ell,s} \choose m}{N_s - A_{w,s}-A_{\ell,s} \choose n-m}$ such samples.
Among these samples, $B_w$ may take values $i=0, 1, \dots, m$.
For a fixed $i$, there are ${A_{w,s} \choose i}{A_{\ell, s} \choose m-i}{N_s - A_{w,s} - A_{\ell,s} \choose n-m}$
samples with $B_w=i$ and $B_\ell = m-i$.

The factor ${N_s - A_{w,s} - A_{\ell,s} \choose n-m}$ counts the number of ways to sample $n-m$ of the
remaining ballots.
If we divide out this factor, we simply count the number of ways to sample ballots
from the group of ballots for $w$ or for $\ell$.
There are ${A_{w,s}+A_{\ell,s} \choose m}$ equally likely samples of size $m$ from
the ballots with either a vote for $w$ or for $\ell$, but not both, 
and of these samples, ${A_{w,s} \choose i}{A_{\ell, s} \choose m-i}$ contain $i$ ballots with a vote for $w$ but not $\ell$.
Therefore, conditional on $B=n$ and $B_w+B_\ell=m$, the probability that $B_w=i$ is

$$\frac{{A_{w,s} \choose i}{A_{\ell, s} \choose m-i}}{{A_{w,s}+A_{\ell,s} \choose m}}.$$

The $p$-value of the simple hypothesis that there are $A_{w,s}$ ballots with
a vote for $w$ but not for $\ell$, $A_{\ell,s}$ ballots with a vote for $\ell$ but not for $w$, 
and $N - A_{w,s} - A_{\ell,s}$ ballots with votes for both $w$ and $\ell$ or neither $w$ nor $\ell$ 
(including undervotes and
invalid ballots) is the sum of these probabilities for events when $B_w - B_\ell \geq nd$.
This event occurs for $B_w \geq \frac{m+nd}{2}$.
Therefore,

\begin{equation}
   \mathbb{P}_{A_{w,s}, A_{\ell,s}, N_s} \left \{ D \geq d \;\vert\; B = n, B_w+B_\ell = m \right \} = 
   \sum_{i=(m+nd)/2}^{\min\{m, A_{w,s}\}} \frac{{A_{w,s} \choose i}{A_{\ell, s} \choose m-i}}{{A_{w,s}+A_{\ell,s} \choose m}}.
\end{equation}


This conditional $p$-value is thus the tail probability of the hypergeometric distribution
with parameters $A_{w,s}$ ``good'' items, $A_{\ell,s}$ ``bad'' items, and a sample of size $m$.
This calculation is numerically stable and fast; tail probabilities of the hypergeometric distribution are available
and well-tested in all standard statistics software.

\note{Check this}
In simulations, the conditional hypergeometric test for an audit showed better performance than
using the tri-hypergeometric distribution directly.
The conditional hypergeometric test tended to correctly reject the null hypothesis more frequently and in one example, its $p$-value was
smaller than the corresponding tri-hypergeometric $p$-value about 80\% of the time.
Furthermore, the tri-hypergeometric distribution is not widely available in statistical packages and its $p$-value calculation involves 
summing over two variables (as opposed to just one in the hypergeometric case), making calculations much slower.
For these reasons, we propose conditioning on both $B=n$ and $B_w+B_\ell=m$ to yield the conditional hypergeometric test.

\subsection{Maximizing the $p$-value over the null set}

The composite null hypothesis does not specify $A_{w,s}$ or $A_{\ell,s}$ separately, only 
that $A_{w,s} - A_{\ell,s} \le c$ for
some fixed, known $c$.
The (conditional) $p$-value of this composite hypothesis for $D=d$ is the maximum $p$-value for all
values $(A_{w,s}, A_{\ell,s})$ that are possible under the null hypothesis,
\begin{equation}
  \max_{A_{w,s}, A_{\ell,s} \in \{0, 1, \ldots, N \}: A_{w,s} - A_{\ell,s} \le c, A_{w,s} + A_{\ell,s} \le N_s}
   \sum_{i = (m+nd)/2}^{\min\{m, A_{w,s}\}} \frac{ {A_{w,s} \choose i}{A_{\ell,s} \choose m-i} }{{A_{w,s} + A_{\ell, s} \choose m}},
\end{equation}
wherever the summand is defined. 
(Equivalently, define ${m \choose k} \equiv 0$ if $k > m$, $k < 0$, or $m \le 0$.)

\subsubsection{Optimizing over the parameter $c$}
The following result enables us to only test hypotheses along the boundary of the null set.

\begin{thm}
Assume that $A_{w,s} > cm/2$, $B_w > B_\ell$, and $c>0$.
Suppose the composite null hypothesis is $N_w - N_\ell \leq c$.
The $p$-value is maximized on the boundary of the null region, i.e. when $N_w - N_\ell = c$.
\end{thm}

\note{This seems to hold for $c<0$ too. The proof doesn't so much matter on $c=0$ and $c>0$, but the difference between the two values. 
Should think about whether this is worth adding, whether the assumptions there are useful.}

\begin{proof}
Without loss of generality, let $c=0$ and assume $A_{w,s}$ is fixed.

The $p$-value $p_0$ for the simple hypothesis that $c=0$ is

\begin{equation}
  p_0 = \sum_{i = (m+nd)/2}^{\min\{m, A_{w,s}\}} \frac{ {A_{w,s} \choose i}{A_{w,s} \choose m-i} }{{2A_{w,s} \choose m}} =  \sum_{i = (m+nd)/2}^{\min\{m, A_{w,s}\}} T_i,
\end{equation}

where $T_i$ is defined as the $i$th term in the summand and $T_i \equiv 0$ for $i > \min\{m, A_{w,s}\}$.

Assume that $c>0$ is given.
The $p$-value $p_c$ for this simple hypothesis is
\begin{align*}
p_c &=   \sum_{i = (m+nd)/2}^{\min\{m, A_{w,s}+c\}} \frac{ {A_{w,s}+c \choose i}{A_{w,s} \choose m-i} }{{2A_{w,s}+c \choose m}}  \\
   &= \sum_{i = (m+nd)/2}^{\min\{m, A_{w,s}\}} T_i \frac{(A_{w,s} + c)(A_{w,s}+c-1)\cdots(A_{w,s}+1)(2A_{w,s}+c-m)\cdots(2A_{w,s}+1) }
   {(A_{w,s}+c-i)\cdots(A_{w,s}+1-i)(2A_{w,s}+c)\cdots(2A_{w,s}+1)} + \Delta,
\end{align*}

where $\Delta \equiv \sum_{i=A_{w,s}+1}^{A_{w,s}+c} \frac{ {A_{w,s}+c \choose i}{A_{w,s} \choose m-i} }{{2A_{w,s}+c \choose m}} $
 is nonzero whenever $A_{w,s} > m-c$.

Terms in the fraction can be simplified: choose the corresponding pairs in the numerator and denominator.
Fractions of the form $(A_{w,s}+c + j)/(A_{w,s}+j-i)$ can be expressed as $1 + i/(A_{w,s}+j-i)$.
Fractions of the form $(2A_{w,s}+k-m)/(2A_{w,s}+k)$ can be expressed as $1 - m/(2A_{w,s}+k)$.
Thus, the $p$-value can be written as 

\begin{align*}
p_c &= \sum_{i = (m+nd)/2}^{\min\{m, A_{w,s}\}} T_i \prod_{j=1}^c \left(1 + \frac{i}{A_{w,s} + j - i}\right)\left(1 - \frac{m}{2A_{w,s}+j}\right)  + \Delta\\
&> \sum_{i = (m+nd)/2}^{\min\{m, A_{w,s}\}} T_i \prod_{j=1}^c \left(1 + \frac{i}{A_{w,s} + c - i}\right)\left(1 - \frac{m}{2A_{w,s}+c}\right)  + \Delta\\
&= \sum_{i = (m+nd)/2}^{\min\{m, A_{w,s}\}} T_i \left[ \left(1 + \frac{i}{A_{w,s} + c - i}\right)\left(1 - \frac{m}{2A_{w,s}+c}\right) \right]^c + \Delta \\
&= \sum_{i = (m+nd)/2}^{\min\{m, A_{w,s}\}} T_i \left[ 1 + \frac{i(2A_{w,s}+c) - m(A_{w,s}-i+c) - im}{(A_{w,s} + c - i)(2A_{w,s}+c)} \right]^c  + \Delta\\
&= \sum_{i = (m+nd)/2}^{\min\{m, A_{w,s}\}} T_i \left[ 1 + \frac{A_{w,s}(2i-m) - c(m-i)}{(A_{w,s} + c - i)(2A_{w,s}+c)} \right]^c  + \Delta
\end{align*}

This can be bounded.
The assumption that $B_w > B_\ell$ implies that $m = B_w+B_\ell \leq 2i \leq 2B_w$ for all $i \leq B_w$.
Thus, $2i-m \geq 1$ and $m-i \geq m-B_w \geq m/2$.
So,

$$p_c > \sum_{i = (m+nd)/2}^{\min\{m, A_{w,s}\}} T_i \left[ 1 + \frac{A_{w,s} - cm/2}{(A_{w,s} + c - i)(2A_{w,s}+c)} \right]^c + \Delta >  \sum_{i = (m+nd)/2}^{\min\{m, A_{w,s}\}} T_i  = p_0$$





\end{proof}


\subsubsection{Optimizing over the parameter $A_{w,s}$}

We have shown empirically (but do not prove) that this tail probability, as a function of $A_{w,s}$,
has a unique maximum at one of the endpoints.
If the empirical result is true, then finding the maximum is trivial;
otherwise, it is a trivial one-dimensional optimization problem to compute the unconditional $p$-value.

\subsection{Conditional testing}
If the conditional tests are always conducted at significance level $\alpha$ or less, i.e., so that
$\mathbb{P} \{\mbox{Type I error} | B = n, B_w+B_\ell = m\} \le \alpha$, then the
overall procedure has significance level $\alpha$ or less:
\begin{eqnarray}
    \mathbb{P} \{\mbox{Type I error}\} &=& \sum_{n=0}^N\sum_{m=0}^{N} \mathbb{P} \{\mbox{Type I error} |  B = n, B_w+B_\ell = m\} \mathbb{P} \{ B = n, B_w+B_\ell = m \} \nonumber \\
       & \le & \sum_{n=0}^N\sum_{m=0}^{N}  \alpha \mathbb{P} \{  B = n, B_w+B_\ell = m \}  =  \alpha.
\end{eqnarray}

In particular, this implies that our conditional hypergeometric test will have the correct risk limit unconditionally.
