\documentclass[12pt]{article}

\bibliographystyle{plainnat}
\usepackage{graphicx}
\usepackage{amssymb,amsmath,color}
\usepackage[breaklinks=true]{hyperref}
\usepackage[round]{natbib}
%\usepackage{amsthm}
%\usepackage{algorithm}
%\usepackage{algorithmicx}
%\usepackage[noend]{algpseudocode}
\newtheorem{thm}{Theorem}
\newtheorem{definition}{Definition}

\newcommand{\beq}{\begin{equation}}
\newcommand{\eeq}{\end{equation}}
\newcommand{\ben}{\begin{enumerate}}
\newcommand{\een}{\end{enumerate}}

\newcommand{\mc}[1]{\ensuremath{\mathcal{#1}}}
\newcommand{\mb}[1]{\ensuremath{\mathbb{#1}}}
\newcommand{\ul}[1]{\ensuremath{\underline{#1}}}
\newcommand{\RM}{\emph{RM}}
\newcommand{\A}{\emph{A}}
\newcommand{\B}{\emph{B}}
\newcommand{\C}{\emph{C}}
\newcommand{\EE}{\bb{E}}
\newcommand{\PP}{\bb{P}}
\newcommand{\bpi}{\bar{\pi}}
\newcommand{\bp}{\bar{p}}

\newcommand{\comment}[1]{}
\newcommand{\note}[1]{\textcolor{red}{\sc #1}}



\title{Preparing to Audit Colorado's 2018 Primaries}

\author{
   Mark Lindeman\\
   Neal McBurnett\\
   Kellie Ottoboni\\
   Ronald L.~Rivest\\
   Philip B.~Stark
}

\date{Draft \today}

\begin{document}
\maketitle


\begin{abstract}
Colorado's current audit software (RLATool) needs to be improved to audit partisan 
primaries in Colorado in 2018, even to draw the sample for the audit:
the current version of RLATool does not allow the user to select the sample size, nor does
it directly allow an unstratified random sample to be drawn across counties.
Similarly, RLATool needs to be modified to recognize that contests can cross jurisdictional
boundaries; currently, it treats every contest as if it were entirely
contained in a single county.
Margins and risk limits apply to entire contests, not to the portion of a contest
included in a county.
Second, to audit a contest that includes voters in ``legacy'' counties 
(counties with voting systems that cannot export cast vote records) 
and voters in counties with newer systems requires new statistics, if one wants to
keep the efficiency of ballot-level comparison audits that the newer systems
afford.
Third, auditing contests that appear only on a subset of ballots can
be made much more efficient if the sample can be drawn from just those ballots
that contain the contest.
While allowing samples to be restricted to ballots reported to contain a particular
is not essential for the June, 2018 primaries, it will be necessary
eventually to make it feasible to audit smaller contests.
\end{abstract}

\section{Introduction}
A risk-limiting audit (RLA) of an election is a procedure that
has a known, pre-specified minimum chance of correcting the electoral outcome if the outcome
is incorrect---that is, if the reported outcome differs from the outcome that a full manual
tabulation of the votes would find. 
RLAs require a durable, voter-verifiable record of voter intent, such as paper ballots,
and they assume that this audit trail is sufficiently complete and accurate that a full hand
tally would show the true electoral outcome.
That assumption is not automatically satisfied: a \emph{compliance audit} is required.

Risk-limiting audits are generally incremental: they examine more ballots, or batches of ballots,
until either (i)~there is strong statistical evidence that a full hand tabulation would confirm the outcome,
or (ii)~the audit has led to a full hand tabulation, the result of which becomes the official
result.

RLAs have been piloted in California, Colorado, and Ohio, and a test of
RLA procedures has been conducted in Arizona.
RLA bills are being drafted or are already under consideration in Virginia, Washington, and other states.
A number of laws have either allowed or mandated risk-limiting post election audits,
including California AB~2023 (Salda\~{n}a), SB~360 (Padilla), and AB~44 (Mullin);
Rhode Island SB~413A and HB~5704A; and Colorado Revised Statutes (CRS)~1-7-515.

CRS~1-7-515 required 
Colorado to implement risk-limiting audits beginning in 2017.
(There are provisions to allow the Secretary of State to exempt some counties.)
The first statewide risk-limiting election audits took place in Colorado in November, 2017. 

Colorado's ``uniform voting system'' program has led
many Colorado counties to purchase (or to plan to purchase) voting systems
that are auditable at the ballot level: those systems export cast vote records (CVRs)
for individual ballots in a manner that allows the corresponding paper ballot to be identified,
and conversely, make it possible to find the CVR corresponding to any
particular paper ballot.
We call counties that have such systems ``CVR'' counties.
It is estimated that by June, 2018, 98.2\% of active Colorado voters will be in CVR counties.
CVR counties can perform ``ballot-level comparisons,'' which are currently the
most efficient approach to risk-limiting audits in that they require examining
fewer ballots than other methods do, when the outcome of the contest under audit 
is in fact correct.

Other counties (``legacy'' or ``non-CVR'' counties) 
have systems that do not allow auditors to check how the system
interpreted voter intent for individual ballots.
Their election results can still be audited, provided their voting systems
create a voter-verifiable paper trail (\emph{e.g.}, voter-marked paper ballots) that is
conserved to ensure that it remains accurate and intact, and organized well enough
to permit ballots to be selected at random.
Pilot audits in California suggest that the most efficient way to audit such systems
is by ``ballot-polling''  
(in contrast to ``batch-level comparisons,'' for example).

There is currently no literature on how to perform risk-limiting audits 
of contests that include CVR counties and non-CVR counties by combining
ballot polling and ballot-level comparisons.
Existing methods would either require all counties to use the lowest
common denominator, ballot-polling (which does not take advantage of the CVRs,
and thus is expected to require more auditing than a method that does take
advantage of the CVRs), or would
require non-CVR counties to perform batch-level comparisons, which were found in
California to be (generally) less efficient than ballot-polling audits.%
\footnote{%
  See~\cite{Rivest-2018-bayesian-tabulation-audits}
  for a different (Bayesian) approach to auditing contests that include both CVR counties
  and non-CVR counties.
  }

This document focuses on near-term requirements for risk-limiting audits in Colorado: June and November 2018.

\subsection{Colorado in June, 2018}
We understand that for June, 2018, Colorado Secretary of State Wayne Williams intends to require
a risk-limiting audit of at least one statewide contest in addition
to a countywide contest in each county.

Auditing efficiency is controlled in part by how well the audit sample can be focused on ballots that
contain the contests under audit.
Some contests are on (essentially) every ballot, for instance the governor's race.
Others, such as mayoral contests, may appear on only a small fraction of ballots cast in
a county.
Partisan primaries---even for statewide office---are somewhere in between,
because in general no single party's primary appears on every ballot cast in the state.
Thus, either we accept a cut to efficiency and sample ballots from counties (or collections of counties)
but keep the simplicity of being able to sample uniformly, or we develop a way to
focus the auditing on the ballots that contain the contest.
The latter requires external information, e.g., from SCORE, as discussed below.

Moreover, party primaries for statewide offices (and perhaps other contests) will
include CVR counties and non-CVR counties, so we need a method to audit
across mixed jurisdictional voting technology.

This report addresses both issues, providing a handful of ways of dealing with heterogeneous
voting technology, varying in efficiency, complexity, and on whom any additional audit burden falls.

\section{Crude (and unpleasant) approaches}

\subsection{Hand count the legacy counties}
The simplest approach to combining legacy counties with CVR counties is to require every
legacy county to do a full hand count of the primaries, and to conduct a 
ballot-level comparison audit in CVR counties, based on contest margins adjusted for
the results of the manual tallies in the CVR counties.
For instance, imagine a contest with two candidates, reported winner $w$ and reported loser $\ell$.
Suppose the total number of reported votes for candidate~$w$ is $V_w$ 
and the total for candidate~$\ell$ is $V_\ell$, so that $V_w > V_\ell$, since 
$w$ is the reported winner.
Suppose that a full manual tally of the votes in the legacy counties shows $V_w'$ votes for $w$ and
$V_\ell'$ votes for $\ell$.
Suppose that a total of $N$ ballots were cast in the CVR counties.
Then the diluted margin for the comparison audit in the CVR counties is 
$[(V_w-V_w')-(V_\ell-V_\ell')]/N$.
This approach is presumably unacceptable because it would require every legacy county to do a full hand
count. 
(But it would provide an incentive for those counties to upgrade their systems
sooner rather than later, and it does not penalize CVR counties for the fact that their
legacy siblings have not yet upgraded.)

\subsection{Subtract error bounds for the legacy counties from vote totals}
If ballot accounting and SCORE can give good upper bounds on the number of ballots cast in
each contest in legacy counties, there are simple upper bounds on the total
possible overstatement error each legacy county could contribute to the overall contest
results; those can be subtracted from the overall margin (as in the previous subsection) and the
remainder of the contests can be audited in CVR counties against the adjusted margins.
For instance, consider a primary that appears on $N$ ballots in a legacy counties.
Suppose that in legacy counties, the overall, statewide contest winner, $w$, is reported to have received $V_w'$ votes, and some loser, $\ell$, is reported to have received $V_\ell'$ votes. 
(Note that $V_\ell'$ could be greater than $V_w'$: $w$ is not necessarily the reported winner in the legacy counties.)
Then the most overstatement error that the county could possibly have in determining whether
$w$ in fact beat $\ell$ is if every reported undervote, invalid vote, or vote for a different candidate, $t$, had 
in fact been a vote for $\ell$ (producing a 1-vote overstatement), and every vote reported for 
$w$ was in fact a vote for $\ell$ (producing a 2-vote overstatement).
The reduction in the margin that would produce is 
$N - V_w' - V_\ell' + 2V_w' = N + V_w' - V_\ell'$ votes.

This approach may be unacceptable for at least two reasons:
first, if the margin is small, it could easily lead to a full hand count in every county.
Second, even if it doesn't lead to a full hand count, it penalizes CVR counties for the
fact that non-CVR counties have not upgraded their systems, because it reduces the
margin in every contest that includes a legacy county. 

\subsection{Treat legacy counties as if every ballot selected from them for audit has a two-vote overstatement}
A third simple-but-pessimistic approach is to sample uniformly from all counties as if one
were performing a ballot-level comparison audit everywhere,  but to 
treat any ballot
selected from a legacy county as a two-vote overstatement.
This approach is probably unacceptable for at least two reasons:
first, if the margin is small, it could easily lead to a full hand count in every county.
Second, even if it doesn't lead to a full hand count, it penalizes CVR counties for the
fact that non-CVR counties have not upgraded their systems, because it will require expanding
the sample (across all counties) every time a ballot is selected from a legacy county.


\section{Variable batch sizes}

A third approach is to perform a comparison audit across all counties, but to use batches consisting
of more than one ballot (batch-level comparisons)
in legacy counties and batches of a single ballot (ballot-level comparisons) in CVR counties.
The constraint here is that the non-CVR counties need to be able to report vote subtotals
for physically identifiable batches.
If a county's voting system can only report subtotals by precinct but 
the county does not sort paper ballots by
precinct, this approach might require revising how the county handles its
paper; we understand that this is the case in many Colorado counties.

That said, many California counties that do not sort vote-by-mail (VBM)
ballots by precinct conduct the statutory 1\% audits by manually retrieving the ballots 
for just those precincts selected for audit from whatever physical batches they happen to be in: 
the situation is identical to that in Colorado.

Another solution is the ``Boulder-style'' batch-level audit, which requires generating 
vote subtotals after each physical batch is scanned, and exporting those subtotals in machine-readable form.
That in turn may require using extra memory cards, repeatedly initializing and deleting tabulation databases,
or other measures that add complexity and opportunity for human error.

While those two approaches are laborious, they would provide a viable short-term solution,
especially combined with information from SCORE to check that the reported batch-level results contain the correct number of ballots for each contest under audit.
Moreover, it does not unduly increase the workload in CVR counties
to compensate for the fact that some other counties have not upgraded their voting
systems.

This kind of variable-batch-size comparison audit approach would require modifying or augmenting
RLATool in several ways: 

\begin{enumerate}

  \item the CVR reporting tool would need to be modified to allow non-CVR counties to
report batch-level results in a manner analogous to how CVR counties report
ballot-level results, or an external tool would need to be provided.

  \item the sampling
algorithm would have to allow sampling batches---and sampling them with unequal probability,
because efficient batch-level audits involve sampling batches with probability proportional
to a bound on the possible overstatement error in the batch.
It would also need to calculate the appropriate sampling probability for each batch (of whatever size).
Again, this could be accommodated using an external tool to draw the sample from legacy counties.

  \item the risk calculations would need to be modified. 
This, too, could be done with external software, with suitable provisions for capturing audit data
from RLATool or directly from legacy counties.
\end{enumerate}

None of these changes is enormous; the mathematics and statistics are already worked out
in published papers, and there is exemplar code for calculating the
batch-level error bounds, drawing the samples with probability proportional to an
error bound, and calculating the attained risk from the sample results.
Indeed, this is the method that was used in several of California's pilot audits,
including the audit in Orange County.
A derivation of a method for comparison audits with variable batch sizes is given below
in section~\ref{sec:comparisonError}.

\section{Stratified ``hybrid'' audits}

Other approaches involve \emph{stratification}: partitioning the cast ballots
into non-overlapping groups and sampling independently from those groups.
One could stratify by county, but in general it is simpler and more efficient
statistically (i.e., results in auditing fewer ballots) to minimize the number of strata.
We consider methods that use two strata: CVR counties and non-CVR counties. 
Collectively, the ballots cast in CVR counties comprise one stratum and the ballots cast in 
legacy counties comprise a second stratum; every ballot cast in the contest is in 
exactly one of the two strata. 
We assume that the samples are drawn from the
two strata independently.

\subsection{Partitioning the total permissible overstatement into strata}
The simplest approach to stratification involves partitioning the risk limit and the tolerable
overstatement error of the tabulation into
two pieces, one for the (pooled) CVR counties and one for the (pooled) non-CVR counties.
Let $V_{w\ell} > 0$ denote the contest-wide margin (in votes) of reported winner 
$w$ over reported loser
$\ell$.
Let $V_{w\ell,s}$ denote the margin (in votes) of reported winner $w$ over reported loser $\ell$
in stratum $s$. 
Note that $V_{w\ell,s}$ might be negative in one stratum.
Let $A_{w\ell}$ denote the margin (in votes)
of reported winner $w$ over reported loser $\ell$ that 
a full hand count of the entire contest would show, that is, the \emph{actual} margin rather
than the \emph{reported} margin.
Reported winner $w$ really beat reported loser $\ell$ if and only if $A_{w\ell} > 0$.
Define $A_{w\ell,s}$ to be the actual margin (in votes) of $w$ over $\ell$ in stratum $s$;
this too may be negative.

Let $\omega_{w\ell,s} \equiv V_{w\ell,s} - A_{w\ell,s}$ be the \emph{overstatement}
of the margin of $w$ over $\ell$ in stratum $s$.
Reported winner $w$ really beat reported loser 
$\ell$ iff $\omega_{w\ell} \equiv \omega_{w\ell,1} + \omega_{w\ell,2} < V_{w\ell}$.

Pick $\lambda_1 \in \Re$ and define $\lambda_2 = 1-\lambda_1$.
If $\omega_{w\ell,1} < \lambda_1 V_{w\ell}$ \emph{and} 
$\omega_{w\ell,2} < \lambda_2 V_{w\ell}$, candidate $w$ really received more votes
than candidate $\ell$.
Some pairs can be ruled out \emph{a priori}, because (for instance) $\omega_{w\ell,s} \in [-2N_s, 2N_s]$,
where $N_s$ is the number of ballots cast in stratum $s$.
There are other simple, sharper bounds, sketched below.

The choice of $\lambda_1$, the strata risk limits $\{\alpha_s\}$, and details of the
audit procedures affect the workload and the overall risk limit.
(See section~\ref{sec:stratumRisk}.)

For ballot-level comparison audits, auditing to ensure that $\omega_{w\ell,s} < \lambda_s V_{w\ell}$
is discussed in section~\ref{sec:comparisonError}; it is a minor modification of the method
embodied in RLATool.

For ballot-polling audits, auditing to ensure that $\omega_{w\ell,s} < \lambda_s V_{w\ell}$ is discussed in section~\ref{sec:ballotPollError}.
Note that this requires a more substantial modification of the standard ballot-polling calculations,
because the standard calculations consider only the fraction of ballots with a vote for either 
$w$ or $\ell$ that contain a vote for $w$, while we need to make an inference about the 
difference between the number of votes for $w$ and the number of votes for $\ell$.
This introduces an additional nuisance parameter, the number of ballots with votes for either
$w$ or $\ell$.

\subsubsection{Combining stratum-level risk limits}\label{sec:stratumRisk}
We audit to test the two hypotheses $\{\omega_{w\ell,s} \ge \lambda_s V_{w\ell}\}_{s=1}^2$, 
independently for the two strata.
If we reject \emph{both} hypotheses, we conclude that the contest outcome is correct;
otherwise, we manually re-tabulate the contest in one or both strata, depending on the
audit rules.
Those rules matter:
generally, the two audits will need to be conducted to smaller risk limits individually than the desired
risk limit for the contest as a whole, unless proceeding to a full hand tabulation in one stratum
automatically triggers a full hand tabulation in the other stratum.

Recall that the samples are drawn independently from the two strata.
Pick $\alpha_1, \alpha_2 \in (0,\alpha)$.
(Below we discuss the choice further.)
Also pick $\lambda_1$.
Then if $\omega_{w\ell,1} < \lambda_1 V_{w\ell}$ and 
$\omega_{w\ell,2} < \lambda_2 V_{w\ell}$,
the outcome is correct.
We audit stratum $s$ to test the hypothesis $\omega_{w\ell,s} \ge \lambda_s V_{w\ell}$ 
with risk limit $\alpha_s$,
as if it were its own election.
We want to know the relationship between those two stratum-level ``risks'' and the 
overall risk that the audit will not correct the outcome if the outcome is wrong.
That depends in part on what we do if the audit in a given stratum leads to a full manual
tally of that stratum.

Consider a few scenarios.
The outcome is certainly correct if both net overstatements are less than their 
respective thresholds. 
For the outcome to be wrong, one or both strata need to have net overstatement
$\omega_{w\ell,s}$
greater than its corresponding threshold $\lambda_s V_{w\ell}$.
If $\omega_{w\ell,1} + \omega_{w\ell,2} \ge V_{w\ell}$, then $\omega_{w\ell,1}\ge \lambda_1V_{w\ell}$
or $\omega_{w\ell,2}\ge \lambda_2V_{w\ell}$, or both.
If the allocated overstatement is exceeded in only one stratum, $s$, then the chance that the 
stratum will be fully hand counted is at least $1-\alpha_s \ge 1- \alpha$.

If both $\omega_{w\ell,1} \ge \lambda_1V_{w\ell}$
and $\omega_{w\ell,2} \ge \lambda_2V_{w\ell}$, then the chance both are fully tabulated is
$1-(1-\alpha_1)(1-\alpha_2)$, since the audit samples in the two strata are independent.
\note{is it not just $(1-\alpha_1)(1-\alpha_2)$?  $P(fail to reject both) = P(fail to reject 1)P(fail to reject 2)$}

What should we do if the audit leads to a full tally in one stratum?
We consider two options.
The simpler is to automatically require a full hand count of the other stratum, 
to set the record straight.
If the audit uses this rule, then we can take $\alpha_1 = \alpha_2 = \alpha$, and the procedure will have
risk limit $\alpha$.

A second approach is to adjust the contest margin for the results of the manual tally in the
fully counted stratum (call the stratum $t$), and continue to audit in the other, 
but against the overall margin, adjusted for the ``known'' tally in the stratum that had 
been counted: we test against the share $V_{w\ell} - A_{w\ell,t} \equiv \lambda_s' V_{w\ell}$, rather than 
against the share $\lambda_s V_{w\ell}$.
Then to reject the null hypothesis in that stratum is to conclude that the overall outcome is still correct.

The statistical wrinkle is that adjusting for the manual tally in the hand-counted stratum 
changes the hypothesis being tested in the remaining
stratum in a way that is itself random:
whether the original null or a new null is tested depends on what the sample in the other stratum
finds.
However, if the hypothesis changes, there's only one value possible for $\lambda_s'$---which
depends on the reported margin and the count in the other stratum---but it's unknown 
until the other stratum count is known.

The solution is through conditioning. 
The samples in the two strata are independent. 
Think of the overall procedure as concluding that the outcome is correct without a full
hand count in both strata if:

\begin{itemize}
   \item the original hypotheses are rejected in both strata (neither stratum is fully hand tabulated)
   \item the hypothesis is not rejected in one stratum $s$; the threshold $\lambda_t$
            is adjusted in the other stratum, and the hypothesis that the overstatement error
            in that stratum is greater than the new limit, $\lambda_t' V_{w\ell}$ is rejected. 
\end{itemize}

Suppose that the outcome is incorrect. 
Then in at least one stratum $s$, $\omega_{w\ell,s}\ge \lambda_s V_{w\ell}$.
The chance that the audit leads to a full hand tabulation in that stratum is at least $1-\alpha_s$,
in which case the ``correct'' count for that stratum will become known.

What happens in the other stratum?
We adjust the margin (or tolerance for overstatement error) and keep auditing.
The value of $\lambda_t'$ is fixed, but unknown before the audit starts.
Consider the conditional probability that a sequential test would reject the hypothesis that the margin is less than $\lambda_t' V_{w\ell}$, given that the other stratum, $s$, is tabulated by hand.
Because the tests in the two strata are independent, that is the unconditional probability
that auditing stratum $t$ against the error tolerance $\lambda_t' V_{w\ell}$ would lead to a full
hand tabulation. 
\note{mention that we are not using a sequential test in this paper?}
If we are using a sequential test in the remaining stratum $t$, the chance that the audit will
go to a full hand count in stratum $t$ is at least $1-\alpha_t$ if the (new) null is true (i.e.,
if the outcome is incorrect). 
The conditioning just delays looking at the value of the test statistic for the new null hypothesis; waiting does not increase the overall chance of incorrectly rejecting the null, 
because the test is legitimately sequential.

Hence, for this procedure, the chance that there will be a full hand count in both strata is at least 
$(1-\alpha_s)(1-\alpha_t)$ if the outcome is incorrect,
even if the probability were zero that both of the original audits would proceed to a full hand count.
The overall risk limit is thus not larger than $1 - (1-\alpha_s)(1-\alpha_t)$.

As an example, suppose we want the overall risk limit to be 5\%. 
If we use a risk limit of 4\% in the no-CVR stratum and a risk limit of 1.04\% in the CVR stratum,
the risk limit will be $1 - 0.96\times 0.9896 < 0.05$.


\subsection{Constraining the total overstatement across strata}
A more statistically efficient approach to ensuring that the overstatement error in the 
two strata does not
exceed the margin is to try to constrain the \emph{sum} of the overstatement errors in the two
strata, rather than constrain the pieces separately.
The null hypothesis $\omega_{w\ell, 1} + \omega_{w\ell, 2} \ge V_{w\ell}$ is true if and only if there exists \textit{some}
values of $\lambda_1$ and $\lambda_2$ such that $\omega_{w\ell, s}\ge \lambda_s V_{w\ell}, s=1, 2$.\footnote{
Namely, letting $\lambda_1 = \frac{\omega_{w\ell, 1}}{\omega_{w\ell, 1}+\omega_{w\ell, 2}}$ satisfies both inequalities.
}
Thus, fixing $\lambda_1$ and $\lambda_2$ at single values and requiring that we reject both stratum-level null hypotheses 
will be inefficient:
there are many ways that the total overstatement could be less than $V_{w\ell}$ (i.e., the alternative hypothesis is true) without
having the overstatement $\omega_{w\ell,s}$ in stratum $s$ less than $\lambda_s V_{w\ell}$, $s = 1, 2$.

To that end, imagine \emph{all} values ways of partitioning the error.
If, for all $(\lambda_1, \lambda_2)$ pairs, we can reject the hypothesis that the 
overstatement error in stratum~1 is greater than or equal to $\lambda_1 V_{w\ell}$ \emph{and} 
the overstatement error in stratum~2 is greater than or equal to $\lambda_2 V_{w\ell}$, then
we can conclude that the outcome is correct.
This is more efficient because it only requires rejecting one of the two stratum-wise null hypotheses,
for all possible $(\lambda_1, \lambda_2)$ pairs,
rather than rejecting \textit{both} null hypotheses for a particular pair.

To test the conjunction hypothesis (i.e., that both of those null hypotheses are true), we use 
Fisher's combining function.
Let $p_s(\lambda_s)$ be the $p$-value of the hypothesis $\omega_{w\ell,s} \ge \lambda_s V_{w\ell}$.
If the null hypothesis that $\omega_{w\ell,1} \ge \lambda_1 V_{w\ell}$ and 
$\omega_{w\ell,2} \ge \lambda_2 V_{w\ell}$ is true, then the combination
\beq
   \chi(\lambda_1, \lambda_2) = -2 \sum_{s=1}^2 \ln p_s(\lambda_s)
\eeq
has a probability distribution that is dominated by the chi-square distribution with 4 degrees
of freedom.\footnote{%
   If the two tests had continuously distributed $p$-values, the distribution would be exactly
   chi-square with four degrees of freedom, but if either $p$-value has atoms when
   the null hypothesis is true, it is in general stochastically smaller.
   This follows from a coupling argument along the lines of Theorem~4.12.3 in \citet{grimmett01}.
}
Fisher's combined statistic will be small when both null hypotheses are true and will be large when
at least one null hypothesis is not true.

Hence, if, for all $\lambda_1$ and $\lambda_2 = 1- \lambda_1$,
the combined statistic $\chi(\lambda_1, \lambda_2)$ is greater than the 
$1-\alpha$ quantile of the chi-square
distribution with 4 degrees of freedom, the audit can stop.

This procedure involves maximizing Fisher's combined statistic over all pairs $(\lambda_1, \lambda_2)$.
The calculation of $p_s(\lambda)$ uses the procedures discussed in 
sections~\ref{sec:comparisonError} and~\ref{sec:ballotPollError}.

\section{Sampling from subcollections} \label{sec:subcollections}

To audit contests that are contained on only a fraction of the ballots cast
in one or more counties efficiently requires the ability to sample from
just those ballots (or, at least, from a subset of all ballots that contains every such ballot).
Because the CVRs cannot be entirely trusted (otherwise, the audit would be superfluous),
we cannot rely on them to determine which ballots contain a given contest.
However, if we have independent knowledge of the number of ballots that
contain a given contest (e.g., from the SCORE system), then there are methods
that allow the sample to be drawn from ballots whose CVRs contain the contest
and still limit the risk rigorously.
See~\citet{benalohEtal11} and \citet{banuelosStark12} for details.

\section{Batch comparison audits of a tolerable overstatement in votes}
\label{sec:comparisonError}

In this section we expand previous comparison auditing work (already embodied in RLATool) to handle two new requirements.  The first allows the specification of the $\lambda$ parameters discussed in section~\ref{sec:hybrid}. The second handles batch-level auditing.

The first requirement requires that we consider auditing in a single stratum to test whether the overstatement of any margin
(in votes) exceeds some fraction $\lambda$ of the overall margin $V_{w\ell}$ between
reported winner $w$ and reported loser $\ell$.
If the stratum contains all the ballots cast in the contest, then for $\lambda = 1$, this 
would confirm the election outcome.
For stratified audits, we might want to test other values of $\lambda$, as described above.

In Colorado, comparison audits have been ballot-level (i.e., batches consisting of a single
ballot). 
This section also addresses the second requirement by deriving a method for batches of arbitrary size, which might be useful
for Colorado to audit contests that include CVR counties and legacy counties.
We keep the \emph{a priori} error bounds tighter than the ``super-simple'' 
method~\cite{stark10d}.
To keep the notation simpler, we consider only a single contest, but the 
MACRO test statistic \cite{stark09c,stark10d} automatically extends the result to 
auditing $C>1$ contests simultaneously.
The derivation is for plurality contests, including ``vote-for-$k$'' plurality contests.
Majority and super-majority contests are a minor 
modification~\cite{stark08a}.\footnote{%
  So are some forms of preferential and approval voting, such as Borda count, and
  proportional representation contests, such as D'Hondt~\cite{starkTeague14}.
  Changes for IRV/STV are more complicated.
}

\subsection{Notation}
\begin{itemize}
    \item  $\mathcal{W}$: the set of reported winners of the contest
    \item  $\mathcal{L}$: the set of reported losers of the contest
    \item  $N_s$ ballots were cast in all in the stratum. (The contest might not appear on all $N_s$ ballots.)
    \item  $P$ ``batches'' of ballots are in stratum $s$. A batch contains one or more ballots. Every ballot in stratum $s$ is in exactly one batch.
    \item  $n_p$: number of ballots in batch $p$. $N_s = \sum_{p=1}^P n_p$.
    \item  $v_{pi} \in \{0, 1\}$: the reported votes for candidate $i$ in batch $p$
    \item  $a_{pi} \in \{0, 1\}$: actual votes for candidate $i$ in batch $p$. 
                  If the contest does not appear on any ballot in batch $p$, then $a_{pi} = 0$.
                  
    \item  $V_{w\ell,s} \equiv \sum_{p=1}^P (v_{pw} - v_{p\ell})$: 
Reported margin in stratum $s$ of reported winner $w \in \mathcal{W}$ over reported loser 
$\ell \in \mathcal{L}$, in votes.

    \item  $V_{w\ell}$: 
Overall reported margin of reported winner $w \in \mathcal{W}$ over reported loser 
$\ell \in \mathcal{L}$, in votes, for the entire contest (not just stratum $s$)

% If I'm not punch drunk, we really did manage to keep generalizations across multiple contests out of 
% the subsequent exposition. However, we still use V
%
%    \item  $V_s$: smallest reported margin in the stratum among all $C$ contests audited using the same sample:
%$V_s \equiv \min_{w \in \mathcal{W}, \ell \in \mathcal{L}} V_{w \ell, s}$
%
%  \item  $V$: smallest reported overall margin among all $C$ contests audited using the same sample:
% $V \equiv \min_{w \in \mathcal{W}, \ell \in \mathcal{L}} V_{w \ell}$
    \item  $V$: smallest reported overall margin between any reported winner and reported loser:
$V \equiv \min_{w \in \mathcal{W}, \ell \in \mathcal{L}} V_{w \ell}$

    \item  $A_{w\ell,s} \equiv \sum_{p=1}^P (a_{pw} - a_{p\ell})$: 
actual margin in the stratum of reported winner $w \in \mathcal{W}$ over reported loser 
$\ell \in \mathcal{L}$, in votes

    \item  $A_{w\ell}$: 
actual margin of reported winner $w \in \mathcal{W}$ over reported loser 
$\ell \in \mathcal{L}$, in votes, for the entire contest (not just in stratum $s$)

\end{itemize}


\subsection{Reduction to maximum relative overstatement}
If the contest is entirely contained in stratum $s$, then
the reported winners of the contest are the actual winners if
$$ 
   \min_{w \in \mathcal{W}, \ell \in \mathcal{L}} A_{w\ell,s} > 0.
$$
Here, we address the case that the contest may include a portion outside the stratum.
To combine independent samples in different strata, it is convenient
to be able to test whether the net overstatement error in a stratum exceeds a given threshold.

Instead of testing that condition directly, we will test a condition that is sufficient 
but not necessary for the inequality to hold, to get a computationally simple test that
is still conservative (i.e., the risk is not larger than its nominal value).

For every winner, loser pair $(w, \ell)$, we want to test
whether the overstatement error exceeds some threshold, generally
one tied to the reported margin between $w$ and $\ell$.
For instance, for a simple stratified audit, we might take the threshold to be
$\lambda_s V_{w\ell}$.

We want to test whether
$$
   \sum_{p=1}^P (v_{pw}-a_{pw} - v_{p\ell} + a_{p\ell})/V_{w\ell} \ge \lambda_s.
$$
The maximum of sums is not larger than the sum of the maxima; that is,
$$
\max_{w \in \mathcal{W},  \ell \in \mathcal{L}}
   \sum_{p=1}^P (v_{pw}-a_{pw} - v_{p\ell} + a_{p\ell})/V_{w\ell}
   \le
  \sum_{p=1}^P  \max_{w \in \mathcal{W},  \ell \in \mathcal{L}} 
  (v_{pw}-a_{pw} - v_{p\ell} + a_{p\ell})/V_{w\ell}.
$$

Define 
$$
  e_p \equiv \max_{w \in \mathcal{W} \ell \in \mathcal{L}} 
     (v_{pw}-a_{pw} - v_{p\ell} + a_{p\ell})/V_{w\ell}.
$$
Then no reported margin is overstated by a fraction $\lambda_s$ or more
if 
$$ 
  E \equiv \sum_{p=1}^P e_p < \lambda_s.
$$

Thus if we can reject the hypothesis $E \ge \lambda_s$, we can conclude that
no pairwise margin was overstated by as much as a fraction $\lambda_s$.

Testing whether $E \ge \lambda_s$ would require a very large sample if we knew nothing at
all about $e_p$ without auditing batch $p$: a single large value of $e_p$ could make
$E$ arbitrarily large.
But there is an \emph{a priori} upper bound for $e_p$.
Whatever the reported votes $v_{pi}$ are in batch~$p$, we can find the
potential values of the actual votes $a_{pi}$ that would make the
error $e_p$ largest, because $a_{pi}$ must be between 0 and $n_p$,
the number of ballots in batch~$p$:
$$
    \frac{v_{pw}-a_{pw} - v_{p\ell} + a_{p\ell}}{V_{w\ell}} \le 
    \frac{v_{pw}- 0 - v_{p\ell} + n_p}{V_{w\ell}}.
$$
Hence,
\begin{equation} \label{eq:uDef}
    e_p \le \max_{w \in \mathcal{W}, \ell \in \mathcal{L}} 
    \frac{v_{pw} - v_{p\ell} + n_p}{V_{w\ell}} \equiv u_p.
\end{equation}

Knowing that $e_p \le u_p$ might let us conclude reliably that $E < \lambda_s$
by examining only a small number of batches---depending on the 
values $\{ u_p\}_{p=1}^P$ and on the values of $\{e_p\}$ for the audited batches.

To make inferences about $E$, it is helpful to work with the \emph{taint} 
$t_p \equiv \frac{e_p}{u_p} \le 1$.
Define $U \equiv \sum_{p=1}^P u_p$.
Suppose we draw batches at random with replacement, with probability $u_p/U$
of drawing batch $p$ in each draw, $p = 1, \ldots, P$.
(Since $u_p \ge 0$, these are all positive numbers, and they sum to 1,
so they define a probability distribution on the $P$ batches.)

Let $T_j$ be the value of $t_p$ for the batch $p$ selected in the $j$th draw.
Then $\{T_j\}_{j=1}^n$ are IID, $\mathbb{P} \{T_j \le 1\} = 1$, and
$$
  \mathbb{E} T_1 = \sum_{p=1}^P \frac{u_p}{U} t_p =
  \frac{1}{U}\sum_{p=1}^P u_p \frac{e_p}{u_p} = 
  \frac{1}{U} \sum_{p=1}^P e_p = E/U.
$$
Thus $E = U \mathbb{E} T_1$. 

So, if we have strong evidence that
$\mathbb{E} T_1 < \lambda_s/U$, we have
strong evidence that $E < \lambda_s$.

This approach can be simplified even further by noting that $u_p$ has
a simple upper bound that does not depend on $v_{pi}$.
At worst, the reported result for batch $p$ shows $n_p$ votes for the 
``least-winning'' apparent winner of the contest with the smallest margin, 
but a hand interpretation would show that all $n_p$ ballots in the batch 
had votes for the runner-up in that contest.
Since $V_{w\ell} \ge V$ and $0 \le v_{pi} \le n_p$,
$$ 
    u_p =  \max_{w \in \mathcal{W}, \ell \in \mathcal{L}} 
    \frac{v_{pw} - v_{p\ell} + n_p}{V_{w\ell}}
    \le  \max_{w \in \mathcal{W}, \ell \in \mathcal{L}} 
    \frac{n_p - 0 + n_p}{V_{w\ell}}
    \le \frac{2n_p}{V}.
$$
Thus if we use $2n_p/V$ in lieu of $u_p$, we still get conservative results.
(We also need to re-define $U$ to be the sum of those upper bounds.)
An intermediate, still conservative approach would be to use this upper bound for
batches that consist of a single ballot, but use the sharper bound (\ref{eq:uDef})
when $n_p > 1$.
Regardless, for the new definition of $u_p$ and $U$,
$\{T_j\}_{j=1}^n$ are IID, $\mathbb{P} \{T_j \le 1\} = 1$,
and
$$
  \mathbb{E} T_1 = \sum_{p=1}^P \frac{u_p}{U} t_p =
  \frac{1}{U}\sum_{p=1}^P u_p \frac{e_p}{u_p} = 
  \frac{1}{U} \sum_{p=1}^P e_p = E/U.
$$

So, if we have evidence that $\mathbb{E} T_1 < \lambda_s/U$, we have evidence that 
$E < \lambda_s$.

\subsection{Testing $\mathbb{E} T_1 \ge \lambda_s/U$}

To test whether $\mathbb{E} T_1 < \lambda_s/U$, there are a variety of methods available.
One particularly ``clean'' sequential method is based on Wald's Sequential Probability
Ratio Test (SPRT) (\cite{wald45}).
Harold Kaplan pointed out this method on a website that no longer exists.
A derivation of this ``Kaplan-Wald'' method is given in~\cite[Appendix A]{starkTeague14};
to apply the method here, take $t = \lambda_s$ in their equation~18.

A different sequential method, the Kaplan-Markov method (also due to Harold Kaplan), 
is given in~\cite{stark09b}.

\section{Ballot-polling audits of a tolerable overstatement in votes}
\label{sec:ballotPollError}

In this section we develop a new method to conduct ballot-polling audits in legacy counties.  
This is a substantial change from existing ballot-polling methods~\cite{lindemanEtal12}, because specifying the $\lambda$ parameters discussed in section~\ref{sec:hybrid}complicates the statistical problem.

Existing ballot-polling methods consider only the fraction of ballots with a vote for either 
$w$ or $\ell$ that contain a vote for $w$,
making the statistical test one for a proportion.
In this case, we need to make an inference about the 
\emph{difference} between the number of votes for $w$ and the number of votes for $\ell$.
This requires dealing with an unknown nuisance parameter not needed in the existing methods, the number of ballots with votes for either $w$ or $\ell$.

We address this nuisance parameter in two ways.
First, our proposal explicitly takes into account ballots with no vote for $w$ or for $\ell$, including ballots for other candidates and invalid ballots,
Second, the risk is maximized over all possible values of the nuisance parameter,
ensuring that the test is conservative.

\subsection{Conditional tri-hypergeometric test}

We consider a single stratum $s$, containing $N_s$ ballots.
Of the $N_s$ ballots,
$A_{w,s}$ have a vote for $w$ but not for $\ell$, $A_{\ell,s}$ have a vote for $\ell$ but not for $w$, and $A_{u,s} = N_s - N_{w,s} - N_{\ell,s}$ have votes for both $w$ and $\ell$ or neither $w$ nor $\ell$, including undervotes and invalid ballots.
We might draw a simple random sample of $n$ ballots ($n$ fixed ahead of time), or we might draw 
sequentially without replacement, so the sample size $B$ could be random.
For instance, the rule for determining $B$ could depend on the data.\footnote{%
   Sampling with replacement leads to simpler arithmetic, but is not as efficient.
}

Regardless, we assume that, conditional on the attained sample size $n$, the ballots are a simple random sample of size $n$ from the $N_s$ ballots in the population.
In the sample, $B_w$ ballots contain a vote for $w$ but not $\ell$, with $B_\ell$ and $B_u$ defined analogously.
The conditional joint distribution of
$(B_w, B_\ell, B_u)$ is tri-hypergeometric: 

\begin{equation}
    \mathbb{P}_{A_{w,s}, A_{\ell,s}} \{ B_w = i, B_\ell = j \vert B=n \} = 
     \frac{ {A_{w,s } \choose i}{A_{\ell,s} \choose j}{N_s - A_{w,s} - A_{\ell,s} \choose n-i-j}}{{N_s \choose n}}.
\end{equation}

Define the diluted sample margin, $D \equiv (B_w - B_\ell)/B$.
We want to test the compound hypothesis $A_{w,s} - A_{\ell,s} \le c$.
The value of $c$ is inferred from the definition
$\omega_{w\ell,s} \equiv V_{w\ell,s} - A_{w\ell,s} = V_{w,s} - V_{\ell,s} - (A_{w,s} -A_{\ell,s})$.
Thus,
$$
    c = V_{w,s} - V_{\ell,s} - \omega_{w\ell,s} = V_{w\ell,s} - \lambda_s V_{w\ell}.
$$
The alternative is the compound hypothesis 
$A_{w,s} - A_{\ell,s} > c$.\footnote{%
    To use Wald's Sequential Probability Ratio Test, we might pick a simple alternative instead, e.g.,
   $A_{w,s} = V_{w,s}$ and $A_{\ell,s} = V_{\ell,s}$, the reported values, provided 
   $V_{w,s} - V_{\ell,s} > c$.
}
Hence, we will reject for large values of $D$.
Conditional on $B=n$, the event $D = (B_w - B_\ell)/B = d$ is the same as $B_w - B_\ell = nd$.\footnote{%
In contrast, the BRAVO ballot-polling
method~\cite{lindemanEtal12}
conditions only on $B_w+B_\ell = m$.
}


The $p$-value of the simple hypothesis that there are $A_{w,s}$ ballots with
a vote for $w$ but not for $\ell$, $A_{\ell,s}$ ballots with a vote for $\ell$ but not for $w$, 
and $N - A_{w,s} - A_{\ell,s}$ ballots with votes for both $w$ and $\ell$ or neither $w$ nor $\ell$ 
(including undervotes and
invalid ballots) is the probability that $B_w - B_\ell \geq nd$.
Therefore,

\begin{equation}
   \mathbb{P}_{A_{w,s}, A_{\ell,s}, N_s} \left \{ D \geq d \;\vert\; B = n\right \} = 
   \sum_{\substack{(i, j) :  i, j\ge 0 \\ i-j \geq nd \\ i+j \leq n}} \frac{ {A_{w,s } \choose i}{A_{\ell,s} \choose j}{N_s - A_{w,s} - A_{\ell,s} \choose n-i-j}}{{N_s \choose n}}.
\end{equation}


%\subsection{Conditional hypergeometric test}
%Another approach is to condition on both the events $B=n$ and $B_w+B_\ell=m$.
%We describe the hypothesis test here, but do not advocate for using it.
%We found that this approach was inefficient in some simulation experiments.
%
%Given $B=n$, all samples of size $n$ from the ballots are equally likely, by hypothesis.
%Hence, in particular, all samples of size $n$ for which $B_w + B_\ell = m$ are equally likely.
%There are ${A_{w,s}+A_{\ell,s} \choose m}{N_s - A_{w,s}-A_{\ell,s} \choose n-m}$ such samples.
%Among these samples, $B_w$ may take values $i=0, 1, \dots, m$.
%For a fixed $i$, there are ${A_{w,s} \choose i}{A_{\ell, s} \choose m-i}{N_s - A_{w,s} - A_{\ell,s} \choose n-m}$
%samples with $B_w=i$ and $B_\ell = m-i$.
%
%The factor ${N_s - A_{w,s} - A_{\ell,s} \choose n-m}$ counts the number of ways to sample $n-m$ of the
%remaining ballots.
%If we divide out this factor, we simply count the number of ways to sample ballots
%from the group of ballots for $w$ or for $\ell$.
%There are ${A_{w,s}+A_{\ell,s} \choose m}$ equally likely samples of size $m$ from
%the ballots with either a vote for $w$ or for $\ell$, but not both, 
%and of these samples, ${A_{w,s} \choose i}{A_{\ell, s} \choose m-i}$ contain $i$ ballots with a vote for $w$ but not $\ell$.
%Therefore, conditional on $B=n$ and $B_w+B_\ell=m$, the probability that $B_w=i$ is
%
%$$\frac{{A_{w,s} \choose i}{A_{\ell, s} \choose m-i}}{{A_{w,s}+A_{\ell,s} \choose m}}.$$
%
%The $p$-value of the simple hypothesis that there are $A_{w,s}$ ballots with
%a vote for $w$ but not for $\ell$, $A_{\ell,s}$ ballots with a vote for $\ell$ but not for $w$, 
%and $N - A_{w,s} - A_{\ell,s}$ ballots with votes for both $w$ and $\ell$ or neither $w$ nor $\ell$ 
%(including undervotes and
%invalid ballots) is the sum of these probabilities for events when $B_w - B_\ell \geq nd$.
%This event occurs for $B_w \geq \frac{m+nd}{2}$.
%Therefore,
%
%\begin{equation}
%   \mathbb{P}_{A_{w,s}, A_{\ell,s}, N_s} \left \{ D \geq d \;\vert\; B = n, B_w+B_\ell = m \right \} = 
%   \sum_{i=(m+nd)/2}^{\min\{m, A_{w,s}\}} \frac{{A_{w,s} \choose i}{A_{\ell, s} \choose m-i}}{{A_{w,s}+A_{\ell,s} \choose m}}.
%\end{equation}
%
%
%This conditional $p$-value is thus the tail probability of the hypergeometric distribution
%with parameters $A_{w,s}$ ``good'' items, $A_{\ell,s}$ ``bad'' items, and a sample of size $m$.
%This calculation is numerically stable and fast; tail probabilities of the hypergeometric distribution are available
%and well-tested in all standard statistics software.

\subsection{Maximizing the $p$-value over the null set}

The composite null hypothesis does not specify $A_{w,s}$ or $A_{\ell,s}$ separately, only 
that $A_{w,s} - A_{\ell,s} \le c$ for
some fixed, known $c$.
Define $\mathcal{S}$ to be the set of pairs $(i, j)$ such that $i, j\ge 0, i-j \ge nd,$ and $ i+j \leq n$.
The (conditional) $p$-value of this composite hypothesis for $D=d$ is the maximum $p$-value for all
values $(A_{w,s}, A_{\ell,s})$ that are possible under the null hypothesis,
\begin{equation}
  \max_{A_{w,s}, A_{\ell,s} \in \{0, 1, \ldots, N \}: A_{w,s} - A_{\ell,s} \le c, A_{w,s} + A_{\ell,s} \le N_s}
   \sum_{\substack{(i, j)\in \mathcal{S}}} \frac{ {A_{w,s } \choose i}{A_{\ell,s} \choose j}{N_s - A_{w,s} - A_{\ell,s} \choose n-i-j}}{{N_s \choose n}},
\end{equation}
wherever the summand is defined. 
(Equivalently, define ${m \choose k} \equiv 0$ if $k > m$, $k < 0$, or $m \le 0$.)

\subsubsection{Optimizing over the parameter $c$}
The following result enables us to only test hypotheses along the boundary of the null set.

\begin{thm}
Assume that $n < A_{w,s}+A_{\ell,s}$.
Suppose the composite null hypothesis is $N_w - N_\ell \leq c$.
The $p$-value is maximized on the boundary of the null region, i.e. when $N_w - N_\ell = c$.
\end{thm}

\begin{proof}
Without loss of generality, let $c=0$ and assume that $A_{u,s}=N_s - A_{w,s} - A_{\ell,s}$ is fixed.
Let $N_{w\ell, s} \equiv A_{w,s}+A_{\ell,s}$ be the fixed, unknown number of ballots for $w$ or for $\ell$ in stratum $s$.
The $p$-value $p_0$ for the simple hypothesis that $c=0$ is

\begin{equation}
  p_0 = \sum_{\substack{(i, j) \in \mathcal{S}}} \frac{ {N_{w\ell, s}/2 \choose i}{N_{w\ell, s}/2 \choose j}{A_{u,s} \choose n-i-j}}{{N_s \choose n}} =  \sum_{\substack{(i, j) \in \mathcal{S}}}T_{ij},
\end{equation}

\noindent where $T_{ij}$ is defined as the $(i, j)$ term in the summand and $T_{ij} \equiv 0$ for pairs $(i, j)$ that don't appear in the summation.

Assume that $c>0$ is given.
The $p$-value $p_c$ for this simple hypothesis is
\begin{align*}
p_c &=   \sum_{\substack{(i, j) \in \mathcal{S}}}  \frac{ {(N_{w\ell, s}+c)/2 \choose i}{(N_{w\ell, s}-c)/2 \choose j}{A_{u,s} \choose n-i-j}}{{N_s \choose n}}  \\
   &=\sum_{\substack{(i, j) \in \mathcal{S}}} T_{ij} \frac{ \frac{N_{w\ell, s}+c}{2}(\frac{N_{w\ell, s}+c}{2}-1)\cdots(\frac{N_{w\ell, s}}{2}+1) (\frac{N_{w\ell, s}-c}{2} -j)\cdots(\frac{N_{w\ell, s}}{2}-1-j) }
   {(\frac{N_{w\ell, s}+c}{2} -i)\cdots(\frac{N_{w\ell, s}}{2}+1-i)(\frac{N_{w\ell, s}-c}{2})\cdots(\frac{N_{w\ell, s}}{2}-1)}.
\end{align*}

Terms in the fraction can be simplified: choose the corresponding pairs in the numerator and denominator.
Fractions of the form $\frac{\frac{N_{w\ell, s}}{2} + a}{\frac{N_{w\ell,s}}{2} + a - i}$ can be expressed as $1 + \frac{i}{\frac{N_{w\ell,s}}{2} + a-i}$.
Fractions of the form $\frac{\frac{N_{w\ell, s}}{2}  - a - j}{\frac{N_{w\ell, s}}{2}  - a}$ can be expressed as $1 - \frac{j}{\frac{N_{w\ell, s}}{2} -a}$.
Thus, the $p$-value can be written as 

\begin{align*}
p_c &= \sum_{\substack{(i, j) \in \mathcal{S}}} T_{ij} \prod_{a=1}^{c/2} \left(1 + \frac{i}{\frac{N_{w\ell,s}}{2} + a-i}\right)\left(1 - \frac{j}{\frac{N_{w\ell, s}}{2} - a}\right) \\
&> \sum_{\substack{(i, j) \in \mathcal{S}}}  T_{ij} \left[ \left(1 + \frac{i}{\frac{N_{w\ell,s}+c}{2} -i}\right)\left(1 - \frac{j}{\frac{N_{w\ell, s}}{2}+1}\right) \right]^{c/2} \\
&= \sum_{\substack{(i, j) \in \mathcal{S}}} T_{ij} \left[ 1 + \frac{\frac{N_{w\ell,s}+c}{2}j + \frac{N_{w\ell,s}}{2}i + i}{(\frac{N_{w\ell,s}+c}{2}-i)(\frac{N_{w\ell,s}}{2}+1)}\right]^{c/2} \\
&> \sum_{\substack{(i, j) \in \mathcal{S}}}  T_{ij}\\
&= p_0
\end{align*}

The last inequality follows from the fact that $i$ and $j$ are nonnegative, and 
that $i < \frac{N_{w\ell,s}+c}{2}$ -- it is a possible outcome under the null hypothesis.




\end{proof}


\subsubsection{Optimizing over the parameter $A_{w,s}$}

We have shown empirically (but do not prove) that this tail probability, as a function of $A_{w,s}$,
has a unique maximum at one of the endpoints when $A_{w,s}$ is either as small or as large as possible,
given $N$, $c$, and the observed sample values $B_w$ and $B_\ell$.
If the empirical result is true, then finding the maximum is trivial;
otherwise, it is a trivial one-dimensional optimization problem to compute the unconditional $p$-value.

\subsection{Conditional testing}
If the conditional tests are always conducted at significance level $\alpha$ or less, i.e., so that
$\mathbb{P} \{\mbox{Type I error} | B = n\} \le \alpha$, then the
overall procedure has significance level $\alpha$ or less:
\begin{eqnarray}
    \mathbb{P} \{\mbox{Type I error}\} &=& \sum_{n=0}^N \{\mbox{Type I error} |  B = n\} \mathbb{P} \{ B = n \} \nonumber \\
       & \le & \sum_{n=0}^N \alpha \mathbb{P} \{  B = n \}  =  \alpha.
\end{eqnarray}

In particular, this implies that our conditional hypergeometric test will have the correct risk limit unconditionally.



\bibliography{./pbsBib}

\end{document}
