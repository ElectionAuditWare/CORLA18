\subsection{Constraining the total overstatement across strata}
A more statistically efficient approach to ensuring that the overstatement error in the 
two strata does not
exceed the margin is to try to constrain the \emph{sum} of the overstatement errors in the two
strata, rather than constrain the pieces separately:
there are many ways that the total overstatement could be less than $V_{w\ell}$ without
having the overstatement $\omega_{w\ell,s}$
in stratum $s$ less than $\lambda_s V_{w\ell}$, $s = 1, 2$.
To that end, imagine \emph{all} values $\lambda_1$.
If, for all such pairs, we can reject the hypothesis that the 
overstatement error in stratum~1 is greater than or equal to $\lambda_1 V_{w\ell}$ \emph{and} 
the overstatement error in stratum~2 is greater than or equal to $\lambda_2 V_{w\ell}$, then
we can conclude that the outcome is correct.

To test the conjunction hypothesis (i.e., that both of those null hypotheses are false), we use 
Fisher's combining function.
Let $p_s(\lambda)$ be the $p$-value of the hypothesis $\omega_{w\ell,s} \ge \lambda V_{w\ell}$.
If the null hypothesis that $\omega_{w\ell,1} \ge \lambda_1 V_{w\ell}$ and 
$\omega_{w\ell,2} \ge \lambda_2 V_{w\ell}$ is true, then the combination
\beq
   \chi(\lambda_1, \lambda_2) = -2 \sum_{s=1}^2 \ln p_s(\lambda_s)
\eeq
has a probability distribution that is dominated by the chi-square distribution with 4 degrees
of freedom.
(If the two tests had continuously distributed $p$-values, the distribution would be exactly
chi-square with four degrees of freedom, but if either $p$-value has atoms when
the null hypothesis is true, it is in general stochastically smaller.
This follows from results in~\citep{???}.)

Hence, if, for all $\lambda_1$ and $\lambda_2 = 1- \lambda_1$,
the combined statistic $\chi(\lambda_1, \lambda_2)$ is greater than the 
$1-\alpha$ quantile of the chi-square
distribution with 4 degrees of freedom, the audit can stop.

The calculation of $p_s(\lambda)$ 
uses the procedures discussed in 
sections~\ref{sec:comparisonError} and~\ref{sec:ballotPollError}.