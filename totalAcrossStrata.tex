\subsection{Constraining the total overstatement across strata}
A more statistically efficient approach to ensuring that the overstatement error in the 
two strata does not
exceed the margin is to try to constrain the \emph{sum} of the overstatement errors in the two
strata, rather than constrain the pieces separately.
The null hypothesis $\omega_{w\ell, 1} + \omega_{w\ell, 2} \ge V_{w\ell}$ is true if and only if there exists \textit{some}
values of $\lambda_1$ and $\lambda_2$ such that $\omega_{w\ell, s}\ge \lambda_s V_{w\ell}, s=1, 2$.\footnote{
Namely, letting $\lambda_1 = \frac{\omega_{w\ell, 1}}{\omega_{w\ell, 1}+\omega_{w\ell, 2}}$ satisfies both inequalities.
}
Thus, fixing $\lambda_1$ and $\lambda_2$ at single values and requiring that we reject both stratum-level null hypotheses 
will be inefficient:
there are many ways that the total overstatement could be less than $V_{w\ell}$ (i.e., the alternative hypothesis is true) without
having the overstatement $\omega_{w\ell,s}$ in stratum $s$ less than $\lambda_s V_{w\ell}$, $s = 1, 2$.

To that end, imagine \emph{all} values ways of partitioning the error.
If, for all $(\lambda_1, \lambda_2)$ pairs, we can reject the hypothesis that the 
overstatement error in stratum~1 is greater than or equal to $\lambda_1 V_{w\ell}$ \emph{and} 
the overstatement error in stratum~2 is greater than or equal to $\lambda_2 V_{w\ell}$, then
we can conclude that the outcome is correct.
This is more efficient because it only requires rejecting one of the two stratum-wise null hypotheses,
for all possible $(\lambda_1, \lambda_2)$ pairs,
rather than rejecting \textit{both} null hypotheses for a particular pair.

To test the conjunction hypothesis (i.e., that both of those null hypotheses are true), we use 
Fisher's combining function.
Let $p_s(\lambda_s)$ be the $p$-value of the hypothesis $\omega_{w\ell,s} \ge \lambda_s V_{w\ell}$.
If the null hypothesis that $\omega_{w\ell,1} \ge \lambda_1 V_{w\ell}$ and 
$\omega_{w\ell,2} \ge \lambda_2 V_{w\ell}$ is true, then the combination
\beq
   \chi(\lambda_1, \lambda_2) = -2 \sum_{s=1}^2 \ln p_s(\lambda_s)
\eeq
has a probability distribution that is dominated by the chi-square distribution with 4 degrees
of freedom.\footnote{%
   If the two tests had continuously distributed $p$-values, the distribution would be exactly
   chi-square with four degrees of freedom, but if either $p$-value has atoms when
   the null hypothesis is true, it is in general stochastically smaller.
   This follows from a coupling argument along the lines of Theorem~4.12.3 in \citet{grimmett01}.
}
Fisher's combined statistic will be small when both null hypotheses are true and will be large when
at least one null hypothesis is not true.

Hence, if, for all $\lambda_1$ and $\lambda_2 = 1- \lambda_1$,
the combined statistic $\chi(\lambda_1, \lambda_2)$ is greater than the 
$1-\alpha$ quantile of the chi-square
distribution with 4 degrees of freedom, the audit can stop.

This procedure involves maximizing Fisher's combined statistic over all pairs $(\lambda_1, \lambda_2)$.
The calculation of $p_s(\lambda)$ uses the procedures discussed in 
sections~\ref{sec:comparisonError} and~\ref{sec:ballotPollError}.
