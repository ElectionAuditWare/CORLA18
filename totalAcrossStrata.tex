The null hypothesis $\omega_{w\ell, 1} + \omega_{w\ell, 2} \ge V_{w\ell}$ is true if and only if there exists \textit{some}
values of $\lambda_1$ and $\lambda_2 = 1-\lambda_1$ such that $\omega_{w\ell, s}\ge \lambda_s V_{w\ell}, s=1, 2$.\footnote{
Set $\lambda_1 = \frac{\omega_{w\ell, 1}}{\omega_{w\ell, 1}+\omega_{w\ell, 2}}$ and $\lambda_2 = 1-\lambda_1$.
}
If, for all $(\lambda_1, \lambda_2)$ pairs, we can reject the hypothesis that the 
overstatement error in stratum~1 is greater than or equal to $\lambda_1 V_{w\ell}$ \emph{and} 
the overstatement error in stratum~2 is greater than or equal to $\lambda_2 V_{w\ell}$, then
we can conclude that the outcome is correct.

To test the conjunction hypothesis that both stratum null hypotheses are true, we use 
Fisher's combining function.
Let $p_s(\lambda_s)$ be the $P$-value of the hypothesis $\omega_{w\ell,s} \ge \lambda_s V_{w\ell}$.
If the null hypothesis that $\omega_{w\ell,1} \ge \lambda_1 V_{w\ell}$ and 
$\omega_{w\ell,2} \ge \lambda_2 V_{w\ell}$ is true, then the combination
\beq
   \chi(\lambda_1, \lambda_2) = -2 \sum_{s=1}^2 \ln p_s(\lambda_s)
\eeq
has a probability distribution that is dominated by the chi-square distribution with 4~degrees
of freedom.\footnote{%
   If the two tests had continuously distributed $p$-values, the distribution would be exactly
   chi-square with four degrees of freedom, but if either $p$-value has atoms when
   the null hypothesis is true, it is in general stochastically smaller.
   This follows from a coupling argument along the lines of Theorem~4.12.3 in \cite{grimmett01}.
}
Fisher's combined statistic will tend to be small when both null hypotheses are true and to be large when
at least one null hypothesis is not true.

If, for all $\lambda_1$ and $\lambda_2 = 1- \lambda_1$, we can reject the conjunction
hypothesis at level $\alpha$, the audit can stop.
The stratified audit thus involves increasing the sample sizes in the two strata until 
either the minimum value Fisher's combined statistic over all pairs $(\lambda_1, \lambda_2)$ is larger than the $1-\alpha$ quantile of the chi-square
distribution with 4 degrees of freedom, or until both strata have been fully hand tabulated.

For Colorado's audits, $p_s(\lambda)$ can be calculated using the methods in 
sections~\ref{sec:comparisonError} and~\ref{sec:ballotPollError}.
In general, $p_s(\lambda)$ could be a $P$-value for the hypothesis
$\omega_{w\ell,s} \ge \lambda_s V_{w\ell}$ from any test procedure (although
if the audit is to be sequential, the tests in the two strata must be sequential tests). 